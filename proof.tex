%!TEX root = main.tex

\section{Beweis des Theorems}

Mit folgendem Beweis endet schließlich dieses Kapitel und somit der Beweis des Haupttheorems.

\begin{proof}[Beweis von Theorem~\ref{thm:haupttheorem}(McMullen)]
	Sei wieder $G=\pi_1(M)$ und ohne Einschränkung $\Delta_G\neq 0$. Definiere weiter \[
	P=\begin{cases}
		b_3(M)+1 &, \text{ falls } b_1(M) \geq 2	\\
		0
	\end{cases}
	\]	
	Nach Corollar~\ref{cor:degreealex} zusammen mit Theorem~\ref{thm:keralexnorm} folgt, dass für alle primitiven $\phi \in H^1(M;\ZZ)$ die in einem offenen Kegel zu einer Seite der Alexander-Einheitskugel liegen, die folgende Gleichheit gilt:
	
		\begin{equation}
		b_1(\ker\phi)= \alex \phi + P \label{eq:alexnorm}
		\end{equation}
	
	Doch nach Lemma~\ref{lem:norm} definieren sowohl Thurston- auch als Alexander-Norm jeweils Halbnormen, folglich ist es keine Einschränkung die Aussage für diese $\phi$ zu zeigen. Man fixiere also dieses gewählte $\phi$ und wähle gemäß Lemma~\ref{lem:minS} eine zu $\phi$ duale eingebettete orientierte Fläche $S$ mit den geforderten Minimalitätseigenschaften. Aus dem Lemma folgt dann, dass wir eine obere Schranke erhalten:
	\begin{equation}
		b_1(\ker\phi) \leq b_1(S) \label{eq:obereSchrankeS}
	\end{equation}
	Ohne Einschränkung habe $S$ nicht-positive Eulercharakteristik, da sonst wegen $b_1(S)= b_0(S) + b_2(S) - \chi(S) \leq 1$, die obere Schranke für $b_1(\ker\phi)\leq 1$  mit \eqref{eq:alexnorm} impliziert, dass die Alexander-Norm verschwindet (man beachte hierbei und im Folgenden die Fälle $b_2(S)=b_3(M) \in \{0,1\}$). Das Theorem folgt nun aus Zusammensetzung der erarbeiteten Gleichungen:
	
		\begin{align*}
		\thur \phi = -\chi(S) &= b_1(S)-b_0(S)-b_1(S)\\
					&\geq b_1(\ker\phi) -1-b_3(M) &&= \alex \phi +P -b_3(M)-1\\
					&&& = \begin{cases}
						\thur \phi &, \text{falls } b_1(M)\geq 2\\
						\thur \phi -b_3(M)-1
					\end{cases}
		\end{align*}
\end{proof} 
	Die einzige Abschätzung entsteht aus der Wahl der Fläche bei Lemma~\ref{lem:minS} durch $b_1(\ker\phi)\leq b_1(S)$. Dass diese bei einer Faserung verschwindet, wird im nächsten Kapitel~\ref{sec:fibrations} über Faserungen diskutiert. Ebenso wird man im nächsten Kapitel einsehen, dass die Abschätzung aus dem Lemma echt ist; oder gleichbedeutend: es existieren durchaus Kohomologieklassen mit $\thur \phi > \alex \phi$.