%!TEX root = main.tex


\section{Vorbereitungen der benötigten Theorien}
\label{sec:vorbereitungen}

\subsection{Über noethersche Moduln und Gruppenringe}
\label{sec:noetherianprinciples}
Im Folgenden sei $R$ stets ein kommutativer Ring mit $1\neq 0$.

Sei $M$ ein $R$-Modul. Dann ist $M$ noethersch, falls jeder Untermodul endlich erzeugt ist, insbesondere $M$ selbst. Ein Ring ist noethersch wenn er es als Modul über sich selbst ist. 

Da für jeden $R$-Modulhomomorphismus $M\to N$, das Bild der Erzeuger von $M$ das Bild des Homomorphismus erzeugt, gilt folgende Proposition:
\begin{prop}
	Sei $M \onto N$ ein $R$-Modulepimorphismus. Dann impliziert die endliche Erzeugbarkeit von $M$ die von $N$. \qed
\end{prop}
\begin{cor}
	Quotienten von endlich erzeugten Moduln sind endlich erzeugt. \qed
\end{cor}
\begin{cor}
\label{cor:noethchaincomplex}
	Sei $C_\bullet$ ein Kettenkomplex von noetherschen $R$-Moduln. Dann ist die Homologie endlich erzeugt über $R$.\qed
\end{cor}

\begin{lem}
\label{lem:epimorphismusnoethersch}
	Ein Epimorphismus von einem noetherschen $R$-Modul $M$ in einen weiteren $R$-Modul $L$ impliziert, das $L$ noethersch ist.
\end{lem}
\begin{proof}
	Jeder Untermodul aus $L$ besitzt ein endliches Erzeugendensystem als Bild endlich vieler Erzeuger des zurückgezogenen Untermoduls.
\end{proof}
\begin{lem}
\label{lem:exaktnoethersch}
	Sei $\seq LMN$ eine kurze exakte Sequenz von $R$-Moduln. Dann ist $M$ noethersch genau dann wenn $L$ und $N$ noethersch sind.
\end{lem}
\begin{proof}
	Sei $M$ noethersch, so ist es mit Lemma~\ref{lem:epimorphismusnoethersch} auch $N$. Weiter ist $L$ isomorph zum noetherschen Untermodul $\ker(M\to N)$. Umgekehrt erhalten wir für jeden Untermodul $M'\subset M$ eine kurze exakte Sequenz von Untermoduln mit den entsprechenden Einschränkungen $\seq {L'}{M'}{N'}$ wobei $L'$ und $N'$ als $R$-Untermoduln endlich erzeugt, somit auch $M'$.
\end{proof}

Sei im Folgenden $M$ noethersch. Dies ist zum Beispiel bei folgender Situation gegeben:
\begin{cor}
	Sei $R$ noethersch und $M$ endlich erzeugter $R$-Modul. Dann ist $M$ noethersch.
\end{cor}
\begin{proof}
	Mit Lemma~\ref{lem:exaktnoethersch} ist die direkte Summe $R^n$ noethersch und mit Lemma~\ref{lem:epimorphismusnoethersch} ist es $M$.
\end{proof}
\begin{lem}
	$M$ ist über $R$ endlich präsentiert. 
\end{lem}
\begin{proof}
	$M$ ist als größter Untermodul endlich erzeugt über $R$ also existiert folgende exakte Sequenz:
	\[
		R^n \to M \to 0
	\]
	da $R^n$ aber noethersch ist und Kerne von Homomorphismen Untermoduln sind, kann die Sequenz auf der linken Seite folgendermaßen ergänzt werden:
	\[
		R^m \to R^n \to M \to 0
	\]
\end{proof}

 Für eine solche freie Auflösung von $M$, die also für noethersche Moduln immer existiert, sind alle Relationen durch $\ker(R^n\to M)$ gegeben. Deswegen nennt man eine darstellende Matrix zur Abbildung $f$ aus $R^m \stackrel f \to R^n \onto M$ auch Präsentationsmatrix. Verschiedene Präsentationsmatrizen unterscheiden sich nur durch die folgenden Operationen, siehe etwa~\cite[Theorem~6.1]{LickorishW.B.Raymond.1997}:
	\textbullet Vertauschen von Zeilen oder Spalten \textbullet Hinzufügen von Einheitsblöcken (direkte Summe mit der Identität) \textbullet Hinzufügen von Nullspalten \textbullet Addieren eines Vielfachen einer Spalte oder Zeile auf eine jeweils andere. Zusammen mit grundlegenden Eigenschaften der Determinante liefert das die Wohldefiniertheit der folgenden Definition:
	\begin{defn}
	Es sei $R^m \to R^n \onto M$ eine endliche Präsentation des $R$-Moduls $M$ mit einer Präsentationsmatrix $X \in R^{n \times m}$. Definiere das $i$-te Elementarideal $E_i(M) \subset R$, als das von den Determinanten der $(n-i)\times (n-i)$-Minoren von $X$ erzeugte Ideal.
    	\end{defn}

        Da sich jede Determinante als Linearkombination von den Determinanten der Minoren schreiben lässt, liefert das eine Inklusionsbeziehung der Elementarideale. Für $n-i\leq 0$ und $n-i > \min(m,n)$ definiert man die Elementarideale gemäß der folgenden aufsteigenden Kette (da ohne Einschränkung $m>n$):
        \[
             0=E_{-1}(A)\subset E_0(A) \subset \cdots \subset E_n(A) = R
         \] 

        \begin{bsp}
        \label{bsp:hauptidealelementarteiler}
            Ist $R$ ein Hauptidealring, so sind die Elementarideale durch den Elementarteilersatz vollständig charakterisiert.
        \end{bsp}
        \begin{bem}
        	Das letzte Beispiel verwendet die Ergiebigkeit der Theorie der endlich erzeugten Moduln über Hauptidealringen. Die Elementarideale definieren nach den obigen Überlegungen Invarianten für endlich erzeugte Moduln über noetherschen Ringen. Dies ist durch eine Art Nachahmung der ersten Schritte des Elementarteilersatzes motiviert, die nur die endliche Präsentation der Moduln verwendet.
        \end{bem}

    Nun ist es später häufig nötig, Berechnungen über dem Gruppenring durchzuführen. Dafür wollen wir uns nun der Definition des Gruppenrings zuwenden und einige --- für Berechnungen essenzielle -- grundlegenden Eigenschaften des ganzzahligen Gruppenrings über einer frei abelschen Gruppe feststellen.

     \begin{defn}[Gruppenring]
    		Sei $G$ eine endlich erzeugte Gruppe. Dann ist der \textit{Gruppenring} definiert als die Menge aller endlichen formalen Summen:
    		\[
    			R[G] = \sum_{g \in G} a_g g, a_g \in \ZZ, g \in G
    		\]
            die durch komponentenweise Addition eine abelsche Gruppe wird und durch die Gruppenverknüpfung und die multiplikative Struktur von $R$ ein Ring mit Eins wird. Die Elemente $g \in G \subset R[G]$ werden als die Gruppenelemente in dem Gruppenring bezeichnet.
    	\end{defn}
    	Wir werden uns ausschließlich mit ganzzahligen oder rationalen Gruppenringen befassen, also mit $R \in \{ \ZZ,\QQ\}$.
    	\begin{bem}
    	\label{bem:gruppenwirkungmodul}
    		Sei also $M$ ein $R$-Modul mit einer Gruppenwirkung von $G$. Dann lässt sich $M$ als $R[G]$-Modul auffassen. Insbesondere werden wir also abelsche Gruppen und $\QQ$-Vektorräume über dem ganzzahligen bzw.\ rationalen Gruppenring betrachten.
    	\end{bem}
        \begin{bem}
            Da der Gruppenring die multiplikative Struktur von der Gruppenverknüpfung erbt, ist $\ZZ [G]$ im Allgemeinen nicht unbedingt kommutativ. Mit diesem Problem, das durch nicht-abelsche Gruppen $G$ entsteht, werden wir uns später beschäftigen müssen.
        \end{bem}
\begin{bsp}
        Falls $G$ eine unendlich zyklische Gruppe mit Erzeuger $t$ ist, lässt sich der Gruppenring über $F$ als $\ZZ[F] = \ZZ[t^{\pm 1}]$, also als Ring der formalen Laurentpolynome in der Variablen $t$ auffassen. 
\end{bsp}
Tatsächlich gilt dies sogar etwas allgemeiner. Dies wollen wir zeigen um Eigenschaften für den ganzzahligen Gruppenring zu entwickeln, die wir aus dem multivariablen Laurentring ableiten. Für diesen gilt nämlich:
\begin{prop}
	Der Laurentring $\ZZ[X_1^{\pm 1},\cdots,X_{n}^{\pm 1}] $ ist noethersch.
\end{prop}
\begin{proof}
	Nach dem Hilbertschen Basissatz ist der Polynomring über $\ZZ$ in endlich vielen Variablen noethersch. Ebenso ist die Lokalisierung eines noetherschen Ringes noethersch, da jedes Ideal in der Lokalisierung Bild eines endlich erzeugten Ideals ist. 
\end{proof}

Da der Laurentring mit $\ZZ$-Koeffizienten kein Hauptidealring ist, ist die vorhergehende Proposition so bedeutend. Die folgende Proposition wird zeigen, dass $\ZZ[F]$ für einen freien $\ZZ$-Modul $F$ auch kein Hauptidealring ist. Man sollte sich also um eine möglichst vollständige Liste guter Eigenschaften von $\ZZ[F]$ bemühen.

\begin{prop}
\label{prop:gruppenringnoethersch}
	Sei $F$ eine freie abelsche Gruppe, also $F\cong \ZZ^b$. Dann existiert ein Isomorphismus zwischen $\ZZ[F]$ und $\ZZ[X_1^{\pm 1},\cdots,X_{b}^{\pm 1}] $. Insbesondere ist $\ZZ[F]$ faktoriell und noethersch.
\end{prop}
\begin{proof}
	Unter Ausnutzung der universellen Eigenschaft des Polynomrings und der Lokalisierung definiert man folgende Abbildung:
	\begin{align*}
			\ZZ[X_1^{\pm 1},\cdots,X_{n}^{\pm 1}] & \to  \ZZ[F]\\
			X_i &\mapsto  f_i
	\end{align*}
	Wenn $f_i$ eine Basis von $F$ darstellt, so erhält man offensichtlich einen Isomorphismus.
\end{proof}

\begin{bem}
\label{bem:einheitengruppenring}
	Unter dieser Identifikation ergibt sich, dass die Einheiten $\ZZ[F]^\times$ genau die Produkte der Gruppenelemente mit den Einheiten von $\ZZ$ sind also $\pm F \subset \ZZ[F]$.
\end{bem}

Der Quotientenkörper von $\ZZ$ besitzt natürlich mehr Einheiten, deswegen ergibt sich für den Laurentring über $\QQ$ die Eigenschaft eines Hauptidealrings.
\begin{lem}
\label{lem:QThauptidealring}
	$\laurent\QQ t$ ist ein Hauptidealring.
\end{lem}
Natürlich könnte man $\QQ$ für dieses Lemma durch jeden beliebigen Körper $\KK$ ersetzen. 
%Allerdings entfällt durch Tensorieren mit einem endlichen Körper nicht jegliche Torsion, entsprechend birgt das mögliche Probleme bei Anwendung des universellen Koeffiziententheorems, da der $\Tor$ Anteil nicht zwangsweise verschwindet und somit gilt Proposition~\ref{prop:tensoring} nicht mehr.
\begin{proof}
	Da $\QQ$ ein Körper ist, ist $\QQ[t]$ ein Hauptidealring. Nun ist allgemeiner die Eigenschaft für einen Ring ein Hauptidealring zu sein abgeschlossen unter Lokalisierung. Der Beweis läuft durch Betrachtung der kanonische Lokalisierungsabbildung:
	\[
		\alpha : \QQ[t] \to \laurent \QQ t
	\]
	Die Ideale in der Lokalisierung sind genau die erweiterten Ideale aus dem ursprünglichen Ring, wegen der Erhaltung durch $\laurent \QQ t \supset I=\alpha_*(\alpha^*(I))$, wobei $\alpha^* = \alpha^{-1}$ die Kontraktion eines Ideals bezeichnet und $\alpha_*(I) = \laurent \QQ t \cdot \alpha(I)$ die Erweiterung eines Ideals. Da aber jedes Element in $\laurent \QQ t$ durch Multiplizieren mit der Einheit $t^N$ im Erzeugnis eines Polynoms liegt, wird ein Erzeuger aus dem Hauptideal $\alpha^*(I)$ auf einen Haupterzeuger in $I$ abgebildet. 
\end{proof}
 In der Tat ist dieser Beweis in dem Fall des Laurentrings doch mehr als einleuchtend, allerdings merkt man sich so auch gut den Beweis für allgemeine Lokalisierungen, indem man bemerkt, dass sich das Multiplizieren mit der Einheit $t^N$ einfach erreichen lässt durch Multiplikation mit Elementen aus der multiplikativen Menge an der lokalisiert wurde --- den neuen Einheiten aus dem lokalisierten Ring.




	% Nach dem Hilbertschen Basissatz ist der Polynomring über $\ZZ$ in endlich vielen Variablen noethersch. Ebenso ist die Lokalisierung eines noetherschen Ringes noethersch, da jedes Ideal in der Lokalisierung Bild eines endlich erzeugten Ideals ist. 
	% Also genügt es eine Surjektion von einem solchen in den Gruppenring zu finden, denn ein surjektiver Ringhomomorhpismus ordnet jedem Ideal im überlagerten Ring ein endlich erzeugtes Ideal im ursprünglichen Ring zu und somit auch ein endliches Erzeugendensystem. Aber ein solcher Epimorphismus von einem --- nach den obigen Überlegungen --- noetherschen Ring ist gegeben durch Ausnutzen der universellen Eigenschaft des Polynomrings und der Lokalisierung:
	% \begin{align*}
	% 		\ZZ[X_1^{\pm 1},\cdots,X_{n}^{\pm 1}] & \to  \ZZ[F]\\
	% 		X_i &\mapsto  f_i
	% \end{align*}


\subsection{Poincaré und Lefschetz Dualität --- eine differentialtopologische Betrachtung}
\label{sec:poinc}

	Die Definition der Thurston-Norm in Kapitel~\todo{refthurston} wird uns eine Invariante auf $H^1(M;\ZZ)$ einer glatten Mannigfaltigkeit $M^m$ liefern, die durch Poincaré bzw.\ Lefschetz Dualität aus einer Invarianten auf $H_{m-1}(M,\partial M;\ZZ)$ hervorgeht. Und tatsächlich werden die folgenden Seiten teilweise in ein wildes Hin- und Herspringen zwischen Homologie und Kohomologie ausarten. Aus diesem Grund sollen hier noch einmal wichtige Grundlagen und Berechnungsmöglichkeiten, die später verwendet werden, erklärt werden.

	Zunächst widmen wir uns einigen Konventionen und Identifikationen.
	        \begin{bem}[Homomorphismen der Abelianisierung]
            \label{bem:fundhomologie}
            Es werden stillschweigend natürliche Identifikationen $H^1(M;\ZZ) \cong \Hom(H_1(M);\ZZ) \cong \Hom(\pi_1(M);\ZZ)$ verwendet. Die erste natürliche Identifikation liefert das universelle Koeffiziententheorem. Auch wenn die erhaltende Sequenz im Allgemeinen nicht natürlich zerfällt, so ist dies im Fall der ersten Kohomologie offensichtlich. Die zweite folgt, da ein Homomorphismus $\phi: H_1(M) \to \ZZ$ gleichbedeutend mit einem Homomorphismus $\hat\phi : \pi_1(M) \to \ZZ$ ist. Dies sieht man wie folgt ein: das Hurewicz Theorem besagt, dass die natürliche Abbildung $\pi_1(M) \to H_1(M)$, die durch Auffassen von Schleifen als singuläre 1-Zykel entsteht, die Abelianisierungsabbildung ist. Diese besitzt die universelle Eigenschaft der Quotientenabbildung zu dem Normalteiler $[\pi_1(M),\pi_1(M)]$. Da $\ZZ$ abelsch ist, folgt $[\pi_1(M),\pi_1(M)] \subset \ker \hat \phi$. Also faktorisiert $\hat \phi$ über eine eindeutige Abbildung $H_1(M) \to \ZZ$. Mit anderen Worten: Das folgende Diagramm kommutiert:
            \[
                \begin{xy}
                    \xymatrix{\pi_1(M)\ar[r]^{\hat \phi} \ar[d] & \ZZ \\
                                H_1(M)\ar[ru]_\phi}
                \end{xy}
            \]
            Dies liefert die eins-zu-eins Beziehung, da $\phi$ nach dem Diagramm offensichtlich ein eindeutiges Element in $\Hom(\pi_1(M);\ZZ)$ definiert. \\ \par
            \noindent\textbf{Konvention.} \emph{In dieser Arbeit bezeichne der Kern einer Klasse $\phi \in H^1(M;\ZZ)$ durchweg den Kern der mit $\phi$ identifizierten Abbildung auf der Fundamentalgruppe.}\\ \par
        \end{bem}
	\begin{bem}[Weitere Identifikationen]
	\label{bem:identifications}
	Sei $N$ eine glatte orientierte $n$-dimensionale Mannigfaltigkeit (mit CW-Struktur).
		Bekanntlicherweise gilt:
		\[
					H_{n-1}(N,\partial N;\ZZ)\cong H^1(N;\ZZ) \cong \Hom(H_1(N),\ZZ) \cong \Hom(\pi_1(N),\ZZ) \cong  [N,S^1]
		\]		
		wobei der
		\begin{enumerate}
			\item -te Isomorphismus bei $\partial N= \emptyset$ nach Poincaré und sonst Lefschetz Dualität,
			\item -te Isomorphismus gilt, da das universelle Koeffiziententheorem eine exakte Sequenz liefert in der diese beiden Terme auftauchen und weiter nur $\Ext(H_0(N),\ZZ) = 0$,
			\item -te Isomorphismus gilt, da nach Hurewicz die Abbildung $\pi_1(N)\to H_1(N)$ die Abelianisierung ist (siehe Bemerkung~\ref{bem:fundhomologie}),
			\item -te Isomorphismus gilt, indem man zeigt, dass jede Abbildung auf dem 1-Skelett $X^1 \subset N \to S^1$ auf $N$ fortgesetzt werden kann. Da das 1-Skelett die Fundamentalgruppe (oder Homologie) erzeugt, also die Inklusion eine surjektive Abbildung $i_*:\pi_1(X^1)\onto\pi_1(N)$ induziert, lässt sich auf diese Weise jeder Homomorphismus $\pi_1(N)\to \ZZ$ auf den Erzeugern aus dem 1-Skelett (genauer möchte man, dass diese Abbildung auf einem maximalen Baum konstant ist) definieren und fortsetzen. 
		\end{enumerate}


	\end{bem}
	\begin{bem}[Untermannigfaltigkeiten als Homologieklassen]
	\label{bem:untermannigfaltigkeitenalshomologie}
		Es sei $S^s$ eine $s$-dimensionale orientierte kompakte Mannigfaltigkeit. Die Orientierung liefert eine eindeutige Fundamentalklasse $[S] \in H_s(S)$ falls $S$ geschlossen ist und sonst $[S,\partial S] \in H_s(S,\partial S)$. Es sei $h:(S^s,\partial S) \into (N^n,\partial N)$ eine Einbettung. Dann definiert diese Einbettung eine Homologieklasse in $H_s(N,\partial N)$ als das Bild der eindeutigen Fundamentalklasse von $S$ unter der induzierten Abbildung $H_s(h)$. Für eine orientierbare kompakte Untermannigfaltigkeit $(S,\partial S) \subset (N,\partial N)$ liefert die Inklusion eine Einbettung und für eine gewählte Orientierung erhält man also eine Homologieklasse $[S]$ oder $[S,\partial S]$ in $H_s(N,\partial N)$.
	\end{bem}
	\begin{thm}
		Sei $f:N^n\to K^m$ eine stetige Abbildung glatter Mannigfaltigkeiten. Dann lässt sich $f$ beliebig nah durch glatte Abbildungen approximieren (wie die Güte der Approximation genau gemessen wird ist hier von keiner Bedeutung), welche homotop zu $f$ sind.
	\end{thm}
	\begin{thm}[Sard und Satz vom regulären Wert]
		Sei $f:N^n \to K^m$ eine glatte Abbildung. Dann liegen die regulären Werte von $f$ dicht in $K$. Das Urbild eines regulären Wertes ist eine abgeschlossene glatte orientierte eigentlich eingebettete Untermannigfaltigkeit in $K$ der Kodimension $m$. Ein regulärer Wert ist ein Punkt aus $K$, bei dessen Urbild $f$ an jedem Punkt einen Epimorphismus auf den Tangentialräumen definiert. 
	\end{thm}
	\begin{thm}[Thom]
	\label{thm:thom}
		Sei $f:N \to S^1$ eine glatte Abbildung. Jedes Urbild eines regulären Wertes definiert nach obigem Theorem und der Bemerkung~\ref{bem:untermannigfaltigkeitenalshomologie} ein Element in $H_{n-1}(N,\partial N;\ZZ)$. Diese ist Poincaré beziehungsweise Lefschetz dual zu $[f]\in H^1(N;\ZZ)$.
	\end{thm}
	Alle diese Aussagen sind bekannte elementare Aussagen der Differentialtopologie und können etwa in \cite{Kreck.2010} oder mit elementaren Methoden in \cite{Hirsch.1991} (bis auf \ref{thm:thom}) nachgelesen werden. 

	Das Zusammentragen aller Ergebnisse bedeutet also: Jeder Homomorphismus $\phi \in H^1(N;\ZZ)$ definiert ein eindeutiges Element in $[N,S^1]$. und umgekehrt liefert jede Abbildung $N\to S^1$ einen induzierten Homomorphismus der Homologiegruppen. Die Differentialtopologie liefert nun die restlichen Schritte. Jede Homotopieklasse aus $[N,S^1]$ enthält einen glatten Repräsentanten. Für diesen existiert ein regulärer Wert in $S^1$. Jedes Urbild eines solchen regulären Wertes ist dann eine 1-kodimensionale orientierte Mannigfaltigkeit (mit Rand, wenn überhaupt, im Rand von $N$ eigentlich eingebettet), welche als Homologieklasse dual zu $\phi$ ist.

	\subsection{Konstruktionen}
	Als explizite Anwendung des letzten Kapitels leiten wir in diesem Kapitel Konstruktionen her, die gewisse Informationen von 1-kodimensionalen Mannigfaltigkeiten beinhalten.
	\begin{lem}
		Wenn $M$ eine glatte orientierte $n$-Mannigfaltigkeit ist und $N\subset M$ eine glatte $k$-dimensionale Untermannigfaltigkeit, dann korrespondieren die Orientierungen von $N$ bijektiv mit den Orientierungen von dem Normalenbündel $\nu (N;M)$ in $M$.
	\end{lem}
	\begin{proof}
		Das Tangentialbündel von $N$ kann als Untervektorraumbündel von $TM$ aufgefasst werden; es ist unabhängig von der Einbettung. Falls $N$ nicht orientierbar ist, so ist es auch $\nu(N;M)$ nicht und umgekehrt. Sei also $N$ orientierbar, dann gilt für alle $x \in N$, dass die Faser $TN_x\oplus \nu(N;M)_x = TM_x$ die Orientierung von $M$ trägt, also $m_1,\cdots,m_n$ eine repräsentierende Basis der eindeutigen Orientierung ist. Also liefert die Lineare Algebra aus einer Orientierung von $N$, definiert durch die Basis des Tangentialraumes $n_1,\cdots,n_k$ eine eindeutige Orientierung von $\nu(N;M)_x$, repräsentiert durch die Basis $\nu_1,\cdots,\nu_{n-k}$, sodass die folgenden Orientierungen übereinstimmen: $[m_1,\cdots,m_n] = [n_1,\cdots,n_k,\nu_1,\cdots,\nu_{n-k}] $
	\end{proof}

	\begin{defn}
		Eine Untermannigfaltigkeit der Kodimension 1 $N\subset M$ heißt zweiseitig, falls eine glatte Einbettung $N \times (-\epsilon,\epsilon) \into M$ existiert, die auf $N \times \{0\} $ mit der Inklusion von $N$ übereinstimmt. Eine solche Einbettung heißt auch zweiseitiger Kragen.
	\end{defn}
	\begin{bem}
	\label{bem:zweiseitigkeit}
		Offensichtlich ist die Zweiseitigkeit einer 1-kodimensionalen Untermannigfaltigkeit gleichbedeutend mit einer Orientierung, da $N \times (-\epsilon,\epsilon)$ diffeomorph zu $\nu(N;M)$ ist. Falls wir im Folgenden von einer orientierten 1-kodimensionale Mannigfaltigkeit $N$ in einer orientierten Mannigfaltigkeit $M$ reden, so soll mit ihrem zweiseitigen Kragen stets ein solcher gemeint sein, unter dessen Urbild jeder Repräsentant $\nu \in [\nu]$ einer Faser im Normalenbündel, der mit der Orientierung von $N$ in dieser Faser die Orientierung von $M$ ergibt, in $N\times (0,\epsilon)$ liegt.
	\end{bem}

Da wir im Folgenden nur 3-Mannigfaltigkeiten betrachten, so gestatten wir jetzt schon die Bequemlichkeit uns darauf zu reduzieren. Die 1-kodimensionalen Untermannigfaltigkeiten sind also diffeomorph zu Flächen.

\label{sec:constr}
Zu einer gegebenen Homologieklasse $  \phi \in H^1(M;\ZZ)$ finden wir nach Kapitel~\ref{sec:poinc}, sowohl den eindeutigen Homomorphismus $\phi:\pi_1(M) \to \ZZ$ als auch die glatte Abbildung $f:M \to S^1$ mit $\pi_1(f)=\phi$. Außerdem finden wir mit Kapitel~\ref{sec:poinc} eine zu $\hat \phi$ duale orientierte Untermannigfaltigkeit $S$ als Urbild von $f$. Mit den folgenden Konstruktionen soll bewiesen werden, dass jede duale orientierte eingebettete Fläche $(S,\partial S) \subset (M,\partial M)$ als Urbild eines regulären Wertes darstellbar ist. Weiter wollen wir häufig die zu $(\ker) \phi$ gehörige Überlagerung betrachten. Wir werden ein Verfahren entwickeln, in welchem wir durch Aufschneiden und Verkleben an $S$ diese Überlagerung erhalten. Genauer wollen wir zeigen: Sowohl durch Aufschneiden an $S$ als auch durch Zurückziehen der universellen Überlagerung von $S^1$ entlang $f$ erhalten wir die Überlagerung zu dem Normalteiler der Fundamentalgruppe $\ker\phi$.

\begin{constr}[Aufschneiden an einer Fläche]
	\label{constr:cut}
	Aus der Überlagerungstheorie ist bekannt, dass zu jeder normalen Untergruppe der Fundamentalgruppe eines hinreichend gut zusammenhängendem Hausdorffraum (insbesondere Mannigfaltigkeiten), auch eine normale Überlagerung existiert, die bis auf Überlagerungsisomorphie eindeutig ist, siehe~\cite[Chapter~1.3]{Hatcher.2002}. Nun definiert aber ein Element $\phi\in H^1(M,\ZZ)$ nach obigen Überlegungen den Normalteiler $\ker\phi\subset \pi_1(M)$. In diesem Sinne nennen wir die Überlagerung zu $\phi$ fortan $M_\phi$. Für spätere Zwecke, wollen wir die dualen Homologieklassen von $\phi$ mit $b_1(\ker\phi)=b_1(M_\phi)$ vergleichen. Da jede duale Homologieklasse nach Kapitel~\ref{sec:poinc} eine Untermanigfaltigkeit als Repräsentanten besitzt, sei also $(S,\partial S) \subset (M,\partial M)$ eine eingebettete orientierte Fläche dual zu $\phi$. Da diese Kodimension~$1$ und eine Orientierung hat, ist sie nach Bemerkung~\ref{bem:zweiseitigkeit} auch zweiseitig. Fixiere also einen zweiseitigen Kragen\footnote{Diese Zweiseitigkeit sei natürlich stets mit der Kompatibilität mit der induzierten Orientierung des Normalenbündels gewählt. Dies wurde auch in Bemerkung~\ref{bem:zweiseitigkeit} verlangt und bedeutet auch Kompatibilität mit der Orientierung.} $h:S\times (-\epsilon,\epsilon) \to M$. Das bedeutet, dass die 3-Mannigfaltigkeit an $S$ "`aufgeschnitten"' werden kann (siehe etwa~\cite[Kapitel~4.2]{Burde.2003}), wobei das Aufschneiden bedeutet, das Komplement der Fläche zu betrachten (das Resultat ist offensichtlich eine Mannigfaltigkeit, jedoch können Eigenschaften wie Kompaktheit oder Randbedingungen entfallen). Will man nun die durch Aufschneiden gewonnene Kopien $(M_i)_{i\in \ZZ}, M_i \cong M-S$ wieder verkleben, erweist sich die Zweiseitigkeit der Fläche als günstig sogar notwendig (sonst würde nur \emph{eine} Kopie von $S$ als Rand entstehen). Der fixierte zweiseitige Kragen liefert nämlich durch $h(S,(-\epsilon,0))$ und $h(S,(0,\epsilon))$ offene Mengen $M_i^-$ und $M_i^+$ in den $M_i$. Durch den strukturerhaltenden Diffeomorphismus $h$ auf sein Bild, können nun $M_i$ und $M_{i+1}$ jeweils entlang $M_i^+$ und $M_{i+1}^-$ verklebt werden --- genauer: $h$ liefert eine Äquivalenzrelation auf der disjunkten Vereinigung 
	\[
		\cdots \sqcup M_{i-1} \sqcup (S \times (-\epsilon,\epsilon)) \sqcup M_i \sqcup  (S \times (-\epsilon,\epsilon)) \sqcup M_{i+1} \sqcup \cdots,
	\]
	sodass der Quotient eine unendlich zyklische Überlagerung mit der offensichtlichen Projektion bildet. Da entlang offener Mengen verklebt wird, also durch die "`Überlappungen"', ist es möglich den gewonnenen Quotienten mit einer differenzierbaren Struktur zu versehen, so dass die Inklusionen der $M_i$ glatte Einbettungen und somit Untermannigfaltigkeiten sind.

	Als weiteren Vorteil dieser expliziten Konstruktion, sieht man explizit die Decktransformationsgruppe. Erzeugt von $t$ ist sie über $t\mapsto 1$ zu $\ZZ$ isomorph. Unter diesem Isomorphismus entspricht $n\in \ZZ$ einer Translation aller $M_i$ um $n$. Es bleibt nur noch zu zeigen, dass diese Überlagerung auch \textit{die} zu $[S]$ gehörige Überlagerung ist, die in dem obigen Sinne dem dualen $\phi$ entspricht (da $S$ immer noch die gewählte Orientierung bzw.\ den gewählten zweiseitigen Kragen trägt). Dies sieht man zum Beispiel ein, indem man sich unter dem Isomorphismus $H^1(M;\ZZ)\cong [M,S^1]$ einen glatten Repräsentanten des Bildes von $\phi$ aussucht. Natürlich existiert so einer nach den Bemerkungen in~\ref{sec:poinc} immer, jedoch soll dieser für den gewünschten Nachweis explizit $S$ als orientiertes Urbild eines regulären Wertes ergeben. Dafür konstruiert man sich aus dem fixierten zweiseitigen Kragen eine Abbildung $f:M\to S^1$, die $S\times 0$ auf $p\in S^1$ abbildet, $M-S \times (-\epsilon,\epsilon)$ konstant auf den antipodalen Punkt von $p$ abbildet und auf $S\times (-\epsilon,\epsilon)$ gemäß der Projektion auf den zweiten Faktor fortgesetzt wird. Bezüglich dieser glatt konstruierten Abbildung $f$ ist der Wert $p$ regulär und $f^{-1}p=S$ mit der richtigen Orientierung ausgestattet. Durch paralleles Aufschneiden von $M$ an $S$, und $S^1$ an $p$ (analog wie oben nur 2 Dimensionen tiefer) erhält man folgendes kommutatives Diagramm von Überlagerungen:
	\[
		\begin{xy}
			\xymatrix{M_\phi \ar[r] \ar[d] &\RR \ar[d]\\
						M \ar[r]& S^1}
		\end{xy}
	\]

	Dieses Diagramm ist ist aber nun ein Pullback Diagramm von glatten Faserbündeln. Also ist die Diffeomorphieklasse von $M_\phi$ eindeutig. 
\end{constr}
Wir können festhalten:
\begin{cor}
\label{cor:verklvertr}
		Die unendlich zyklische Überlagerung, die durch Aufschneiden und Verkleben an einer zu $\phi$ dualen Fläche entsteht, entspricht der normalen Untergruppe $ \ker\phi$.
\end{cor}
\begin{cor}
\label{cor:preimage}
	Jede zu $\phi$ duale Fläche kann als orientiertes Urbild eines regulären Wertes einer glatten Abbildung $M\to S^1$ dargestellt werden, mit $H_1(M \to S^1)=\phi$.
\end{cor}
Mit einer zu $\phi$ dualen Fläche ist eine eigentlich eingebettete orientierte 1-kodimensionale Untermannigfaltigkeit gemeint.

\begin{constr}[Graph einer orientierten 1-kodimensionalen Untermannigfaltigkeit]
	\label{constr:graph}
	Sei $S$ ein beliebiger zu $\phi \in H^1(M,\ZZ)$ eigentlich eingebetteter, orientierter Repräsentant. Wir haben gesehen, dass mit obiger Konstruktion eine Abbildung $f:M\to S^1$ entsteht mit $f^{-1}p=S$. Diese Abbildung soll in dieser Konstruktion über einen Graphen faktorisiert werden.

	Bezeichne $S=S_1\sqcup \cdots \sqcup S_n$ und $M-S = M_1 \sqcup \cdots \sqcup M_m$ die Zusammenhangskomponenten von $S$ bzw. $M-S$. Betrachte nun den gerichteten Graphen $G$, dessen Knoten bijektiv den Komponenten $M_i$ entsprechen und dessen Kanten aus den Komponenten $S_i$ mit ihrer Orientierung hervorgehen, also ein Graph mit $m$ Knoten und $n$ Kanten, wobei eine Kante von einem Knoten zu einem anderen verläuft, wenn ihre assoziierten Komponenten $M_i, M_j$ durch das entsprechende Flächenstück von $S$ getrennt werden, sodass die Komponenten das zweiseitige Flächenstück an der negativen beziehungsweise positiven Seite berühren, je nachdem ob die Kante vom assoziierten Knoten aus oder eingeht.

	Mit genau diesen zweiseitigen Umgebungen der Flächenkomponenten ist es möglich, ähnlich wie oben eine Abbildung $M \to G$ zu definieren, welche die Assoziierungen respektiert. Dafür betrachte man den zweiseitigen Kragen auf den Komponenten von $S$:
	\[
		\sqcup (S_i \times (-\epsilon,\epsilon)) \stackrel = \longrightarrow S \times (-\epsilon,\epsilon) \into M
	\]
	Dann existiert analog zur obigen Konstruktion die Quotientenabbildung $q:M\to G$ auf den Graph, durch Kollabieren der $M_i\cap (M -S \times(-\epsilon,\epsilon))$ auf ihre Knoten und Projektion von $(-\epsilon,\epsilon)$ auf das Innere der Kanten des Graphen. Man betrachte außerdem $G \to S^1$ die Abbildung die jede Kante entsprechend ihrer Richtung, also orientierungserhaltend einmal um die Sphäre $S^1 = I/\partial I$ abbildet, sodass die Knoten nach $[\partial I]$ abgebildet werden. Außerdem seien die Mittelpunkte der Kanten das Urbild von dem antipodalen Punkt von $[I/\partial I]$ der genau $p\in S^1$ heißt. Bezüglich der Komposition der beiden Abbildungen $M \to S^1$, ist nun $\phi$ das Bild des Erzeugers von $H^1(S^1)$ unter der Rückziehung auf der Kohomologie, da $M \to S^1$ eine zu $S$ duale Kohomologieklasse definiert (da $S=(M\to S^1)^{-1}p$). Sei $f$ nach wie vor die Abbildung aus der letzten Konstruktion. Wir erhalten:
	
	\begin{align}
		\begin{xy}
				\xymatrix{M \ar[r] \ar@/^1pc/[rr]^f & G \ar[r] & S^1 }
			\end{xy}
		\label{eq:graphlift}
	\end{align}
	Da $M$ zusammenhängend ist, ist es auch $G$.
\end{constr}

Im Folgenden wollen wir häufig den Spezialfall betrachten, dass $\phi: H_1(M) \to \ZZ$ surjektiv ist. 
\begin{defn}
	Ein solches $\phi \in H^1(M;\ZZ)$ heißt primitiv.
\end{defn}

\subsection{Normen und Halbnormen auf freien $\ZZ$-Moduln und ihre Fortsetzungen}
    
    Ziel der folgenden Kapitel wird es sein, einer Diffeomorphieklasse von Mannigfaltigkeiten eine Halbnorm und somit eine Invariante zuzuordnen. Genauer gesagt werden es sogar mehrere Halbnormen sein. Die Halbnormen werden wir zunächst auf der ersten Kohomologie $H^1(M;\ZZ)$ definieren. Die erste Kohomologie eines kompakten topologischen Raumes ist stets torsionsfrei, denn es gilt: $H^1(M;\ZZ) = \Hom(H_1(M;\ZZ);\ZZ) = \Hom(H_1(M;\ZZ)/T;\ZZ) \cong \ZZ^n$, wobei $T$ der Torsionsanteil der abelschen Gruppe $H_1(M;\ZZ)$ ist. Es wurde $\Hom(T;\ZZ)=0$ und $H_1(M;\ZZ) \cong H_1(M;\ZZ)/T \oplus T$ verwendet.

    \begin{defn}
    	Eine ganzzahlige Halbnorm ist eine Abbildung $||\cdot||: \ZZ^n \to \ZZ$  die Skalarmultiplikativität und Subadditivität erfüllt. Also für alle $\lambda \in \ZZ$ und $v,w\in \ZZ^n$ soll $||\lambda v||=\vert\lambda\vert~||v||$ und die Dreiecksungleichung $||v+w||\le ||v||+||w||$ gelten.
    \end{defn}
    Es folgt natürlich aus der Skalarmultiplikativität, dass die $0$ trivial von der Halbnorm ausgewertet wird, da $||0|| = 0||0||$. Gilt auch die Umkehrung, also $||v||=0, v\in \ZZ^n \Rightarrow v=0$, so nennen wir $||\cdot ||$ eine ganzzahlige Norm. Analog seien rationale (Halb-)Normen und reelle (Halb-)Normen definiert. 
    \begin{lem}
    \label{lem:fortsetzungnorm}
    	Eine ganzzahlige Halbnorm lässt sich auf eine rationale Halbnorm fortsetzen. Eine ganzzahlige oder rationale Halbnorm lässt sich auf eine reelle Halbnorm fortsetzen. Das gleiche gilt für Normen.
    \end{lem}
    \begin{proof}
    	Will man $||\cdot||:  \ZZ^n \to \ZZ$ rational fortsetzen so bemerkt man, dass die Einbettung $\ZZ^n \into \QQ^n$ die Norm bereits durch die geforderte Skalarmultiplikativität fortgesetzt wird. Sobald man einen Wert für einen Punkt auf einem 1-dimensionalen Unterraum hat, so hat man ihn bereits für den ganzen Unterraum. Jede Gerade durch den Ursprung in $\QQ^n$ schneidet einen und somit unendlich viele Gitterpunkte: Für jedes $q \in \QQ^n$ ist bereits $aq \in \ZZ^n$ für ein großes $a\in \ZZ$. Diese lineare Fortsetzung ist offensichtlich eine Halbnorm sogar eine Norm, falls $||\cdot||$ es auf $\ZZ^n$ war.

    	Sei nun $||\cdot||: \QQ^n \to \QQ$ eine Halbnorm. Dann ist diese insbesondere eine konvexe Funktion, also auf jedem Kompaktum Lipschitz-stetig und somit auf jedem Kompaktum $K\cap \QQ^n, K\subset \RR^n$ in eindeutiger Weise stetig auf $K$ fortsetzbar --- es folgt die Existenz einer rellen Fortsetzung.
    \end{proof}



    \begin{bem}[Duale Vektorraumhalbnorm]
        Eine (Halb-)Norm auf einem Vektorraum, liefert stets eine (Halb-)Norm auf seinem Dualraum. Diese ist für einen Vektorraum $(V,|\cdot|)$ auf dem Dualraum $(V^*,||\cdot||)$ definiert durch:
        \[
             ||\alpha||= \sup_{\{v\in V, |v|=1\}} |\alpha v|
         \] 
         Entsprechend lässt sich eine Halbnorm auf $H^1(M;\RR)$ auf $H_1(M;\RR)$ durch den natürlichen Isomorphismus auf den Bidualraum definieren. Weiter lässt sich eine solche Halbnorm der ersten Kohomologie bei kompakten orientierbaren 3-Mannigfaltigkeiten mittels der Dualitätssätze auf $H_2(M,\partial M;\RR)$ übertragen, die wiederum durch Vektorraumdualität eine Halbnorm für $H^2(M,\partial M;\RR)$ liefert.
    \end{bem}


    \subsection{Die betrachteten Räume}
    	Nun noch eine letzte Generalvoraussetzung, die ab sofort verlangt wird:

       \emph{Im Folgenden sei $M$ stets eine kompakte, orientierbare, zusammenhängende $3$-dimen-sionale Mannigfaltigkeit. Falls diese einen Rand hat, sei er diffeomorph zu einer disjunkten Vereinigung von Tori.}

        Um die Hilfsmittel zu erweitern oder Extremfälle auszuschließen, arbeitet man häufig in der Kategorie der stückweise linearen (PL, \textit{piecewise linear}) oder der differenzierbaren Mannigfaltigkeiten. Sieht man sich beispielsweise Knoten an, also Einbettungen $S^1 \into S^3$, so stellt man fest, dass diese besonders ausgeartet aussehen können, allerdings nicht, falls es sich bei der Einbettung um eine PL oder eine differenzierbare Einbettung handelt. Ähnlich kann man in diesen Kategorien raumfüllende Kurven vermeiden. Allerdings stellte sich im Studium der $3$-Mannigfaltigkeiten heraus, dass die Situation sich erheblich von der in höheren Dimensionen unterscheidet. Beispielsweise ist nicht jede Gruppe als Fundamentalgruppe einer $3$-Mannigfaltigkeit realisierbar, eine Einschränkung liefert etwa Theorem~\ref{thm:keralexnorm}. Außerdem lässt sich die Tatsache ergiebig nutzen, dass in Dimension~$3$ keine Unterscheidung der topologischen, der PL und der differenzierbaren Kategorie nötig ist: nach Moise's Theorem~\cite{Moise.1952}, besitzt jede $3$-Mannigfaltigkeit sowohl eine eindeutige PL als auch eine eindeutige differenzierbare Struktur. Das bedeutet, um Aussagen für $3$-Mannigfaltigkeiten zu zeigen, kann man sich (fast) beliebig in diesen Kategorien hin- und herbewegen um das Resultat am Ende für alle zu erhalten. Um diese beiden Beispiele hervorzuheben, lässt sich bemerken, dass der $\RR^4$ überabzählbar viele unterschiedliche differenzierbare Strukturen besitzt und auch jede Gruppe realisierbar als Fundamentalgruppe einer $4$-Mannigfaltigkeit ist.

        Diese Arbeit wird sich jedoch konsistent in der $C^\infty$-Kategorie bewegen. Der vorherige Absatz dient also lediglich der Betonung, dass dies keine Einschränkung bedeutet. 