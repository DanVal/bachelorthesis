%!TEX root = main.tex

\section{Persönliche Notizen während der Erstellung}
	Sei $m(G)/m(\ker \phi)m(G)$ die algebraische Beschreibung des Alexander Moduls als $\ZZ[F]=\ZZ[G]/m(\ker \phi)$-Modul. Dann dieser Quotient endl erz denn

	Alex und Thurston Norm sind wirklich Normen 

	Warum nimmt Mcmullen die Homologie relativ allen Lifts vom Basispunkt?  !!

	Einschränkung: Mannigfaltigkeit glatt?

	Im Beweis wo $b_1(G)=1$ gezeigt wird, ist es nicht auch möglich, die Argumentation über die Fundamentalgruppe wegzulassen und stattdessen zu zeigen:
	\begin{itemize}
		\item G ist zsh
		\item $\delta_+(M_i) = \delta_-(M_i)=1 \implies G$ ist Mannigfaltigkeit
		\item $G$ ist kompakt
	\end{itemize}
	$\implies$ mit Klassifikation von 1-Mft ist $G$ ein Kreis.

	Andersrum: wenn $G$ vom Homotopietyp ein Kreis ist (also zsh und $\pi_1=\ZZ$), dann kann nur noch $\delta_\pm(M_i)=0$ passieren

	Eigenschaften der Tnorm : $kerx$ ist ein linearer Unterraum. und x $x$ ist auf den Nebenklassen von kerx konstant! und 1-ball

	zweiseitiger KRagen

	noch uct erwähnen

	obda fibration indivisible (flp.pdf)



	vielleicht noch Knoten: duale Fläche als Urbild ist Seifertfläche
	bis jetzt nur: Seifertfläche => dual


	fragen an Fr Hamenstädt
	\begin{itemize}
		\item CW endlich Struktur (Kpt argument oder Morse Argument)
		\item  berandende Fläche nullhomolog wenn mehr Radkomponenten
		\item 
	\end{itemize}