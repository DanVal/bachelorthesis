%!TEX root = main.tex

\section{Algebra}

Tatsächlich könnte man diese Bachelorarbeit fast ausschließlich algebraisch verstehen --- wenigstens den Teil über die Alexanderinvarianten. Natürlich ist dieser Arbeit aus dem Bereich der Topologie, ein Geschmack oder die Illusion gegeben, man arbeite nur mit direkten Eigenschaften der Räume. Bei der Thurstonnorm ist dies tatsächlich gegeben, durch die Deftinition einar Funktion auf einem Vektorraum mittels geometrischer Eigenschaften der Mannigfaltigkeit. Jedoch handelt es sich bei den Alexanderinvarianten, insbesondere dem Alexander Polynom, lediglich um Invarianten einer abelschen Gruppe. Durch Anwendung auf die Fundamentalgruppe, erhält man also Invarianten von Räumen. Natürlich verliert das Alexander Polynom von 3-Mannigfaltigkeiten durch diese "Faktorisierung" nicht allzu viel an Reiz, da die Fundamentalgruppe eine recht starke Invariante von 3-Mannigfaltigkeiten ist, siehe z.B. Ergebnisse von \todo{warum Fundamentalgruppe stark ist} . Im Folgenden sollen nun kurz die Alexander Invarianten verallgemeinert von Gruppen definiert werden und ein paar Zusammenhänge zu dem obigen Fall

Es lässt sich der Alexander Modul definieren als (entspricht strikt genommen nicht der obigen Definition, hier enthält er einen freien Summand mehr, also betrachte $E_1$).

\begin{defn}[Alexander Modul, algebraisch]
	Für eine endlich erzeugte Gruppe $G$ und einen Homomorphismus $\phi \in \Hom(G,F) \cong \Hom(H_1(G),F) \cong \Hom(ab(G),F)$, wobei $F$ eine freie abelsche Gruppe ist und $ab(G)$ die natürliche Quotientenabbildung in den maximalen freien abelschen Quotienten von $G$. Dann ist der Alexander Modul definiert als
	\[
		A_\phi(G)=m(G)
	\]
\end{defn}