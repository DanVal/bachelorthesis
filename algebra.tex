%!TEX root = main.tex

\section{Algebra}

Tatsächlich könnte man diese Bachelorarbeit fast ausschließlich algebraisch verstehen --- wenigstens den Teil über die Alexanderinvarianten. Natürlich ist dieser Arbeit aus dem Bereich der Topologie, ein Geschmack oder die Illusion gegeben, man arbeite nur mit direkten Eigenschaften der Räume. Bei der Thurstonnorm ist dies tatsächlich gegeben, durch die Deftinition einar Funktion auf einem Vektorraum mittels geometrischer Eigenschaften der Mannigfaltigkeit. Jedoch handelt es sich bei den Alexanderinvarianten, insbesondere dem Alexander Polynom, lediglich um Invarianten einer abelschen Gruppe. Durch Anwendung auf die Fundamentalgruppe, erhält man also Invarianten von Räumen. Natürlich verliert das Alexander Polynom von 3-Mannigfaltigkeiten durch diese "Faktorisierung" nicht allzu viel an Reiz, da die Fundamentalgruppe eine recht starke Invariante von 3-Mannigfaltigkeiten ist, siehe z.B. Ergebnisse von \todo{warum Fundamentalgruppe stark ist} . Im Folgenden sollen nun kurz die Alexander Invarianten verallgemeinert von Gruppen definiert werden und ein paar Zusammenhänge zu dem obigen Fall

Es lässt sich der Alexander Modul definieren als (entspricht strikt genommen nicht der obigen Definition, hier enthält er einen freien Summand mehr, also betrachte $E_1$).

\begin{defn}[Alexander Modul, algebraisch]
	Für eine endlich erzeugte Gruppe $G$ und einen Homomorphismus $\phi \in \Hom(G,F) \cong \Hom(H_1(G),F) \cong \Hom(ab(G),F)$, wobei $F$ eine freie abelsche Gruppe ist und $ab(G)$ die natürliche Quotientenabbildung in den maximalen freien abelschen Quotienten von $G$. Dann ist der Alexander Modul definiert als
	\[
		A_\phi(G)=m(G)
	\]
\end{defn}

Dies mach alles interesanter denn es bedeutet dass für eine gegebene Fundamentalgruppe die Alexander Invarianten schon berechenbar sind (also eigentlich Invarianten der Fundamentalgruppe sind), jedoch lassen sich diese auch über Eigenschaften und Berechnungen von 3-Mft berechnen und so ohne Berechnung der Fundamentalgruppe möglich sind. 


Herleitung der algebraischen Idee: nun stellt sich die Frage, ob die Alexander Invarianten durch die Struktur der Mannigfaltigkeit bestimmt sind, oder schon aus der Fundamentalgruppe berechenbar. Sei dazu $G$ die Fundamentalgruppe von $M$ und $F=ab(G)$ der maximale freie abelsche Quotient, dessen Rang durch $b_1(G)$ eindeutig bestimmt ist. Sei außerdem $\alpha:G\to F$ die Quotientenabbildung. Dann erhält man einen Gruppenisomorphismus $p'_*: H_1(\hat M) \to \ker\alpha/[\ker\alpha,\ker\alpha]$ der durch die Projektion der Überlagerung zu $\alpha$ induziert wird. Ziel wäre es zu zeigen, dass diese Abbildung auch einen Isomorphismus von $\ZZ[F]$ Moduln liefert. Daraus würden natürlich gleiche Präsentationen folgen, die zu gleichen Elementaridealen führen und so fort. Hierfür benötigt man jedoch überhaupt eine $\ZZ[F]$-Modul Struktur auf dem Quotienten $G'/G''$ für $G'=\ker\alpha, G''=[G',G']$. Diese soll zunächst definiert werden:\\
Die Abbildung $G \to G/[G,G] \to F$ ist kanonisch. Also seien $g_1,\cdots,g_n,n=b_1(G)$ Elemente die auf eine Basis von $F$ abgebildet werden. Diese definieren dann innere Automorphismen von $G'/G''$ durch 
\[
	\hat t_i(x) = \bar{ g_i} \bar{ x}\overline{ g_i^{-1}} = \overline{{} g_i x g_i^{-1}}
\]
Diese sind unabhängig der gewählten $g_i$. 
\begin{bem}
 	Das ist eine Form der expliziten Konstruktion des Falles, wenn man von einer induzierten Operation einer zerfällenden kurzen exakten Sequenz redet:
 	\[
 		\seq {G'/G''} {G/G''} {G/G'}
 	\]
 	Da $F= G/G'$ frei abelsch ist, zerfällt diese Sequenz und man erhält eine Abbildung $F \to \Aut(G'/G'')$ genauer gesagt, eine Gruppenwirkung auf $G'/G''$ durch Konjugation unter der Identifikation $G'/G''\into G/G''$ mit zurückgezogenen Elementen aus $F$. Dies ist wohldefiniert da das Bild von $G'/G''$ einem Kern entspricht, also Normalteiler ist. Offensichtlich stimmt diese Operation mit der obigen überein.
 \end{bem} 
 Mit $\alpha(g_i)=t_i$ als ein Element der Basis von $F$, ist die Gruppenwirkung von $F$ auf $H_1(\hat M)$, die durch die Decktransformationen induzierte $t_i\gamma = t_{i*}\gamma,t_i \in F \cong D(M), \gamma \in H_1(\hat M)$. Also ist nur zu zeigen, dass folgendes Diagramm kommutiert:
\[
	\begin{xy}
		\xymatrix{H_1(\hat M) \ar[d]_{t_i} \ar[r]^{p'_*} & G'/G'' \ar[d]^{\hat t_i}\\
		H_1(\hat M)  \ar[r]^{p'_*} & G'/G'' }
	\end{xy}
\]
und somit die Operationen verträglich sind. Davon überzeugt man sich, indem man die Wirkung von $t_i$ näher betrachtet: nach Hurewicz lässt sich ein Zykel aus $H^1(\hat M)$ als Schleife darstellen. Die Decktransformation $t_i$ schiebt diese Schleife nun auf eine Schleife, die homolog ist zu der Kompositionskonjugation mit einer zu einem Weg gelifteten Schleife $\tau_i, [\tau_i] \in \pi_1(M)$, welche die Decktransformation $t_i$ erzeugt. \\
Also lassen sich alle Definitionen der Alexander Invarianten analog zu Abschnitt~\ref{sec:defs}, auf endlich erzeugten Gruppen definieren, wobei der Alexander Modul $G'/G''$ als Gruppenring-Modul über $G/G'=F$ aufgefasst wird. Fast! Die Definition der Elementarideale setzt noch voraus, dass dieser Modul endlich erzeugt ist.


Es stellt sich sogar heraus, dass in den meisten Fällen auch die Thurston Norm, aus der Fundamentalgruppe berechnet werden kann, indem man getwistete Alexander Polynome verwendet, die mit ähnlichen Methoden wie oben, allein aus der Fundamentalgruppe gewonnen werden können.