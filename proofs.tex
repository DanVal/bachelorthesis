%!TEX root = main.tex
\section{Beweis des Theorems}

Bevor die Wahl einer dualen Fläche spezifiziert wird, sodass die zu zeigende Abschätzung daraus folgen wird, widmet sich der folgende Teil der zu betrachtenden Überlagerung bezüglich einer Homologieklasse. Will man mit einer Überlagerung Berechnungen anstellen, so sollte man sie möglichst gut kennen. Deswegen folgt nun eine explizite Konstruktion, die sich im späteren Beweis als hilfreich herausstellen wird.

\begin{bem}[Aufschneiden an einer Fläche]
	\label{constr:cut}
	Aus der Überlagerungstheorie ist bekannt, dass zu jeder normalen Untergruppe der Fundamentalgruppe eines einigermaßen gut zusammenhängendem Hausdorffraum (insbesondere Mannigfaltigkeiten), auch eine normale Überlagerung existiert, die bis auf Überlagerungsisomorphie eindeutig ist. Nun definiert aber ein primitives Element $\phi\in H^1(M,\ZZ)$ per Definition einen Homomorphismus auf der ersten Homologiegruppe, aber durch die universelle Eigenschaft der Abelianisierungsabbildung zusammen mit Hurewicz bedeutet das, das eine eindeutige Abbildung $\hat \phi$ existiert, die über $\phi$ faktorisiert, mit anderen Worten das folgende Diagramm kommutiert:
	\[
		\begin{xy}
			\xymatrix{\pi_1(M)\ar[r] \ar[d] & \ZZ \\
						H_1(M)\ar[ru]}
		\end{xy}
	\]
	Dieses $\phi$ liefert also die eindeutige normale Untergruppe $[\pi_1(M),\pi_1(M)]\subset\ker\hat\phi \subset \pi_1(M)$, welche wiederrum eine Überlagerung definiert, die fortan $M_\phi$ genannt wird. Da die Thurston Norm eigentlich eine Halbnorm auf der zweiten Homologie einer 3-Mannigfaltigkeit definiert und über Poincaré bzw.~Lefschetz Dualität nach $H_1(M)$ übertragen wird, stellt sich die Frage nach einer Abhängigkeit der Überlagerung von einer dualen Fläche. Sei also $(S,\partial S) \subset (M\partial M)$ eine eingebettete orientierte Fläche. Da diese Kodimension $1$ hat, ist sie auch zweiseitig, hat also eine Umgebung $U$, sodass ein Homöomorphismus $S\times (-\epsilon,\epsilon) \to U$ existiert dessen Einschränkung auf $S\times \{0\}$ die Inklusion ist. Das bedeutet, dass die 3-Mannigfaltigkeit an $S$ "aufgeschnitten" werden kann (siehe etwa~\cite{Burde2003}), wobei das Aufschneiden bedeutet, das Komplement der Fläche zu betrachten (das Resultat ist offensichtlich eine Mannigfaltigkeit, jedoch können Eigenschaften wie Kompaktheit oder Randbedingungen entfallen). Will man nun durch Aufschneiden gewonnene Kopien $(M_i)_{i\in \ZZ}, M_i \cong M-S$ wieder verkleben, erweist sich die Zweiseitigkeit der Fläche als günstig. Eine zuvor fixierte zweiseitige Abbildung $h: S\times (-\epsilon,\epsilon) \to M$ liefert nämlich durch $h(S,(-\epsilon,0))$ und $h(S,(0,\epsilon))$ offene Mengen $M_i^-$ und $M_i^+$ in den $M_i$. Durch den strukturerhaltenden Diffeomorphismus $h$, können nun $M_i$ und $M_{i+1}$ jeweils entlang $M_i^+$ und $M_{i+1}^-$ verklebt werden --- genauer: $h$ liefert eine Äquivalenzrelation auf der disjunkten Vereinigung 
	\[
		\cdots \sqcup M_{i-1} \sqcup (S \times (-\epsilon,\epsilon)) \sqcup M_i \sqcup  (S \times (-\epsilon,\epsilon)) \sqcup M_{i+1} \sqcup \cdots,
	\]
	sodass der Quotient eine unendlich zyklische Überlagerung mit der offensichtlichen Projektion bildet. Ein weiterer Vorteil dieser exlpliziten Konstruktion ist es, die Decktransformationsgruppe zu sehen. Sie ist durch einen Erzeuger $t$ über $t\mapsto 1$ zu $\ZZ$ isomorph und unter diesem Isomorphismus entspricht $n\in \ZZ$ einer Translation aller $M_i$ um $n$. Es bleibt nur noch zu zeigen, dass diese Überlagerung auch \textit{die} zu $S$ gehörige Überlagerung ist, die in dem obigen Sinne dem dualen $\phi$ entspricht. Dies sieht man ein, indem man sich (mithilfe der Zweiseitigkeit) unter dem Isomorphismus $H^1(M;\ZZ)\cong [M,S^1]$ einen glatten Repräsentanten des Bildes von $\phi$ aussucht (\todo{ein solcher existiert immer}). \todo{eventuelle Umstrukturierung: Konstruktionen, zuerst der Graph (und somit als Korollar, dass jede duale orientierte (somit zweiseitige) Fläche als Urbild dargestellt werden kann), dann die zyklische Überlagerung}. Bekanntlicherweise finden wir eine solche Abbildung $f$ und ein $p \in S^1$, sodass $f^{-1}(p)$ genau die orientierte Fläche ist. Durch paralleles Aufschneiden von $M$ an $S$ und $S^1$ and $p$ erhält man folgendes kommutatives Diagramm von Überlagerungen:
	\[
		\begin{xy}
			\xymatrix{M_\phi \ar[r] \ar[d] &\RR \ar[d]\\
						M \ar[r]& S^1}
		\end{xy}
	\]
	Andererseits ist dies auch ein Pullback Diagramm, da aber der Pullback einer Überlagerung bezüglich $[f]$ bis auf Homöomorphie eindeutig ist, folgt dass die unendlich zyklische Überlagerung durch Aufschneiden und Verkleben zu $\ker(f_*:\pi_1(M)\to\pi_1(S^1)) = ker\phi$ ist.
\end{bem}

\begin{bem}[Fläche als Urbild eines regulären Wertes]
	\label{constr:presurf}
	Sei $S$ ein beliebiger zu $\phi \in H^1(M,\ZZ)$ eingebetteter, orientierbarer Repräsentant. Dann existiert eine Abbildung $M \to S^1$, die einen Erzeuger auf $\phi$ zurückzieht, und ein regulärer Wert $p \in S^1$, so dass $S=f^{-1}(p)$. 
\end{bem}

Nun wollen wir uns eine besondere Art von Flächen anschauen, nämlich die Repräsentanten der zu $\phi$ dualen Klasse, bei denen $\chi_-$ minimal ist. Das immer ein Repräsentat existiert, der gewissen Eigenschaften genügt, sichert der folgende Satz:

\begin{lem}
	\label{lem:minS}
	Sei $\phi \in \Hom (\pi_1(M),\ZZ)$ ein primitives Element dessen Kern endlichen Rang hat. Dann existiert eine zusammenhängende Thurstonnorm-minimierende Fläche $(S,\partial S) \subset (M,\partial M)$ mit $\phi \mapsto [S]$ unter Poincaré Dualität und $b_2(S)=b_3(M)$, so dass folgende Abschätzung erfüllt ist:
	\[
	b_1(S) \leq b_1(ker(\phi))
	\]
\end{lem}
\begin{proof}
	Wähle unter allen Thurstonnorm-minimierenden Flächen eine orientierte Fläche $S$ mit einer geringsten Anzahl an Zusammenhangskomponenten.\\
	\textit{Behauptung: Diese Fläche ist zusammenhängend}\\
	Bezeichne $S=S_1\sqcup \cdots \sqcup S_n$ und $M-S = M_1 \sqcup \cdots \sqcup M_m$ die Zusammenhangskomponenten von $S$ bzw. $M-S$. Betrachte nun den gerichteten Graphen $G$ dessen Knoten bijektiv den Komponenten $M_i$ entsprechen und dessen Kanten aus den Komponenten $S_i$ mit ihrer Orientierung hervorgehen, also ein Graph mit $m$ Knoten und $n$ Kanten, wobei eine Kante von einem Knoten zu einem anderen verläuft, wenn ihre assozierten Komponenten $M_i, M_j$ durch das entsprechende Flächenstück von $S$ getrennt werden, wobei die Komponenten das Flächenstück in der negativen beziehungsweise positiven Umgebung berühren je nachdem ob die Kante vom assozierten Knoten aus oder eingeht. Mit genau diesen zweiseitigen Umgebungen der Flächenkomponenten ist es möglich sich eine Abbildung $M \to G$ zu definieren, welche die Assozierungen respektiert. Sei außerdem $G \to S^1$ die Abbildung die jede Kante entsprechend ihrer Richtung einmal um die Sphäre abbildet und die Knoten auf einen ausgezeichneten Punkt. Bezüglich der Komposition der beiden Abbildungen $M \to S^1$, ist nun $\phi$ das Bild des Erzeugers von $H^1(S^1)$ unter der Rückziehung auf der Kohomologie, da $M \to S^1$ eine zu $S$ duale Kohomologieklasse definiert. Genauer gesagt, kann die Abbildung offensichtlich so gewählt werden, dass das Urbild eines Punktes (verschieden dem ausgezeichneten) einer zu $S$ homologen Fläche ist, also $M \to S^1$ unter der bekannten Bijektion $H^1(M,\ZZ) \cong [M,S^1]$ genau $\phi$ entspricht. Unter diesen Identifikationen, ist klar, dass $G$ homöomorph zu einem Kreis ist. Dafür betrachte man folgendes Diagramm von Pullbacks von Überlagerungen:

	\[
	 	\begin{xy}
	 		\xymatrix{
	 			M_\phi \ar[r] \ar[d] & G_\phi \ar[d] \ar[r] & \RR \ar[d]\\
	 			M \ar[r] & G \ar[r] &S^1
	 		}
	 	\end{xy}
	 \] 

	 Es folgt unmittelbar, dass diese Überlagerungen zyklisch sind, also unendlich zyklische Decktransformationsgruppen haben. Jedes Element in $\pi_1(G_\phi)$ ist homotop zu einem Lift einer Schleife  aus $\pi_1(G)$ ($G$ erbt den Zusammenhang von $M$, deswegen die Vernachlässigung des Basispunktes). Unter Annahme einer Kompatibilitätsvorraussetzung dieser Überlagerungen durch Pullbacks mit den Überlagerungen durch Aufschneiden an dualen Flächen (deswegen die suggerierende Schreibweise $M_\phi$), entsteht $G_\phi$ durch "Aufschneiden an den Knoten", also liftet jede Schleife aus $G$ trivial. Folglich ist $G_\phi$ einfach zusammenhängend, überlagert $G$ also universell. Somit ist $\pi_1(G)= \ZZ$. Also ist $G$ vom Homototyp ein Kreis. Da $G$ aber auch die Kompaktheit von $M$ erbt, ist nur noch die Existenz von Knoten ohne ein- oder ausgehende Kanten auszuschließen. Diese ist aber durch die Minimalitätseigenschaft im Bezug auf die Komponenten der gewählten Fläche ausgeschlossen, da unter den Identifikationen alle Kanten $S_i, i\in I$(nur aus- bzw.~eingehend) von einem solchen Knoten $M_i$, beranden:
	 \[
	 	(M_i\cup \sqcup_{i\in I} S_i, \pm \sqcup_{i\in I} S_i) \implies [\sqcup_{i\in I}S_i]=0
	 \]
	 dies würde eine Fläche mit $|I|$ weniger Komponenten liefern:
	 \[
	 	[S]=[\sqcup_{i\not \in I} S_i\bigsqcup \sqcup_{i\in I} S_i] = [\sqcup_{i\not \in I} S_i \bigsqcup \sqcup_{i\in I} S_i] = [\sqcup_{i\not \in I}S_i]
	 \]
	 Betrachtet man nun die induzierte Abbildung auf der Homologie $(G\to S^1)_*$, so ist diese ein Isomorphismus, da $\phi$ primitiv ist. Also besitzt $G$ nur eine Kante und die Fläche $S$ ist zusammenhängend.\\
	 Die nächste Gleichheit, dass der Rang auf den Top-Homologien von $S$ und $M$ übereinstimmen, hängt von der Existenz eines Randes ab. Da $(S,\partial S) \subset (M,\partial M)$ folgt aus $\partial S \neq \emptyset$ direkt $b_2(S)=b_3(M)=0$. Falls $S$ aber leeren Rand hat, gilt $b_2(S)=1$ und es muss $b_3(M)=1$ gezeigt werden. Äquivalent dazu wird die Existenz eines Randes von $M$ widerlegt:\\
	 Nach Annahme existeren nur Tori als Randkomponenten. Sei $T \subset \partial M$ eine solche Randkomponente. Da $S$ keinen Rand hat, also $T$ nicht berührt, enthält die unendlich zyklische Überlagerung $M_\phi$ auch unendlich zyklisch viele Kopien von $T$ als Randkomponenten. Unter Verwendung der obigen Konstruktion \ref{constr:cut}, liftet $T$ in jedes $M_i$. Im Folgenden soll die Notation $\hat M_i$ für die (wieder) kompakte (Unter-)Mannigfaltigkeit verwendet werden, die durch die Einschränkung auf den Quotienten von $(S\times (-\epsilon,0]) \sqcup M_i \sqcup (S\times[0,\epsilon))$  \todo{komische Richtung ggf. oben ändern} entsteht. Zusammen mit der langen exakten Sequenz für eine kompakte orienierbare Mannigfaltigkeit $(N,\partial N)$
	\[
	 \begin{xy}
	 	\xymatrix{
	 	H_2(N;\QQ) \ar[r]&  H_2(N,\partial N ; \QQ) \ar[r]& H_1(\partial N;\QQ) \ar[r]& H_1(N;\QQ) \\
	 	& H^1(N;\QQ)\ar[u]&&}
	 \end{xy}
	 \] 
	 und der daraus folgenden Abschätzung
	 \[
	 	b_1(\partial N) = \dim(\im \delta) + \dim(\im i_*) \leq  2b_1(N)
	 \]
	 erhält man für jede kompakte zusammenhängende Untermannigfaltigkeit der Form
	 \[
	  	\bigcup_{i\in I} \hat M_i \subset M_\phi
	  \]
	  die Abschätzung:
	  \[
	   	b_1(\bigcup_{i\in I} \hat M_i)\geq \frac{1}{2}b_1(\partial \bigcup_{i\in I} \hat M_i) \geq \frac{1}{2}b_1(\sqcup_{i \in I}T) = |I|
	  \]


	  Nun folgt aber aus der Mayer Vietoris Sequenz (für entsprechende offene Umgebungen) die exakte Sequenz:
	  \[
	  	\cdots\to H_1(S\sqcup S;\QQ) \to H_1(\bigcup_{i \in I} \hat M_i;\QQ) \oplus H_1(M_\phi -\bigcup_{i \in I} \hat M_i;\QQ) \to H_1(M_\phi;\QQ) \to 0
	  \]
	  und somit
	  \[
	  	b_1(M_\phi)= b_1(\bigcup_{i \in I} \hat M_i)+b_1(M_\phi -\bigcup_{i \in I} \hat M_i)-b_1(S\sqcup S) \geq |I| -2b_1(S)
	  \]
	  Da aber $b_1(M_\phi)$ nach Voraussetzung endlich ist und $|I|$ beliebig groß werden kann, folgt also dass der Rand von $M_\phi$ keine Tori enthält und somit leer ist.\\
	  Um nun noch die Abschätzung $b_1(M) \leq b_1(S)$ zu zeigen, wird erneut die Konstruktion der Überlagerung durch Aufschneiden und Verkleben zur Hilfe genommen. Da $\ker\phi \tensor \QQ \cong H_1(M_\phi;\QQ)$ nach Voraussetzung ein endlich erzeugter $\QQ$-Vektorraum ist, wird $H_1(M_\phi;\QQ)$ von einem kompakten Teilraum, etwa der Untermannigfaltigkeit $\hat M_1 \cup \cdots \cup \hat M_k \into M_\phi$ und somit auch $\hat M_{k+1} \cup \cdots \cup \hat M_{2k}\into M_\phi$, erzeugt (die Inklusionen erzeugen Epimorphismen auf der ersten Homologie). Mit diesem Wissen liefert die folgende exakte Sequenz die gesuchte Abschätzung:
	  \[
	  	\cdots \to H_1(S;\QQ) \to H_1(\bigcup_{i\leq 0}\hat M_i;\QQ) \oplus H_1(\bigcup_{i>0} \hat M_i;\QQ) \onto H_1(M_\phi;\QQ)
	  \]
\end{proof}

Da die erste Kohomologie der Überlagerung natürlich bessere Chancen hat, als Modul über dem Gruppenring $\ZZ[t^{\pm1}$ endlich erzeugt zu sein, stellt sich die Frage, ob, wie und warum es sinnvoll oder möglich wäre das eben bewiesene für diesen Fall zu verallgemeinern. Dies wird später mit Hilfe der weiteren Lemmas in diskutiert. Nun vergleicht das vorangegangene Lemma also die Thurston Norm einer Kohomologieklasse mit dem Rang ihres Kerns. Wie letzerer mit der Alexander Norm in Verbindung steht, stellt folgendes Lemma (vgl. Assertion 4) fest:
\begin{lem}
	\label{lem:charPol}
	Es sei wieder $\phi \in H^1(M;\ZZ)$ eine primitive Klasse und $\ker \phi \tensor \QQ$ ein endlich dimensionaler Vektorraum. Weiter sei $t$ ein Erzeuger der Decktransformationsgruppe von $M_\phi$, sodass wie in \ref{wirkung:gruppenring} $H^1(M_\phi)$ als Gruppenring Modul aufgefasst werden kann, wobei der Gruppenring kanonisch mit $\ZZ[t^{\pm 1}]$ identifiziert wird. Dann ist das Elementarideal $E_0(H^1(M_\phi))\subset \QQ[t^{\pm1}]$ bezüglich einer $\QQ[t^{\pm 1}]$-Präsentation ein Hauptideal.\\
	Insbesondere erzeugt für $b_1(M)=1$ das Alexander Polynom den Alexander Modul.
\end{lem}
In dem Beweis wollen wir nutzen, dass $\laurent\QQ t  $ ein Hauptidealring ist. Deswegen folgendes Hilfsmittel:
\begin{lem}
	$\laurent\QQ t$ ist ein Hauptidealring.
\end{lem}
Natürlich könnte man $\QQ$ durch jeden beliebigen Körper $\KK$ ersetzen.
\begin{proof}
	Da $\QQ$ ein Körper ist, ist $\QQ[t]$ ein Hauptidealring. Es existiert eine kanonische Lokalisierungsabbildung:
	\[
		\alpha : \QQ[t] \to \laurent \QQ t
	\]
	Die Ideale in der Lokalisierung sind Bilder der Ideale aus dem ursprünglichen Ring, wegen der Erhaltung durch $I=\alpha_*(\alpha^*(I))$, wobei $\alpha_*,\alpha^*$ die \todo{induzierten Abbildung auf der Menge der Ideale sind}. Also ist das Ideal $I\subset \laurent \QQ t$ das Bild eines Hauptideals. Da aber jedes Element in $\laurent \QQ t$ durch Multiplizieren mit der Einheit $t$ im Erzeugnis eines Polynoms liegt, wird ein Erzeuger aus dem Hauptideal $\alpha^*(I)$ auf einen Haupterzeuger in $I$ abgebildet.	
\end{proof}
\begin{proof}
	Da $H_1(M_\phi,\QQ)$ ein endlich dimensionaler Vektorraum ist, werden durch den Erzeuger $t$ des Quotienten $\pi_1(M)/\ker(\phi) \cong \ZZ$ der Decktransformationen Relationen auf den Basiselementen $x_1,\cdots,x_n$ eingeführt:	
	\begin{align*}
		t_*x_1 &= \sum a_i^1 x_i \\
				&\vdots \\
		t_*x_n &= \sum a_i^n x_i
	\end{align*}
	Diese Gleichungen definieren genau die Matrix des Automorphismus von Vektorräumen $t_* \in \Aut (H_1(M_\phi;\QQ)$, die also als Spalten die $a^i$ hat. Durch subtrahieren der obigen Gleichungen, erhält man eine formale Matrix der Form $A-tI$. Diese Matrix ist aber gleichzeitig die Präsentationsmatrix der freien Auflösung:
	\[
		\begin{xy}
			\xymatrix@L+5pt{\laurent \QQ t ^n \ar[r]_{e_r\mapsto \sum a_i^rx_i-tx_r} & \laurent \QQ t ^n \ar[r] & H_1(M_\phi;\QQ) \ar[r] &0\\}
		\end{xy}
	\]
	Entsprechend ist die Determinante dieser Matrix das Elementarideal bezüglich $\laurent \QQ t$, $E_0(M_\phi)=det(A-tI)=\chi(A) \subset \laurent \QQ t $. 
\end{proof}

Dieses Ergebnis liefert nun einen Zusammenhang zwischen den Thurston Norm-minimierenden Flächen, deren erste Bettizahl nach Lemma~\ref{lem:minS} immer mit oberer Schranke $b_1(\ker\phi)$ gewählt werden kann, und der Alexander Norm:
\begin{cor}
	Sei $\phi \in H^1(M;\ZZ)$ eine primitive Klasse. Dann gilt:
	\[
		\dim(\ker\phi \tensor \QQ) = \Grad(\Delta_\phi) 
	\]
\end{cor}
\begin{proof}
	Nach dem vorherigen Lemma gilt $\dim \Delta_\phi = n+1$ und somit %$H_1(\ker \phi;\QQ) \cong \QQ^n \cong \laurent \QQ t / (\Delta_\phi) \cong \laurent \ZZ t /(\Delta_\phi) \tensor \QQ$ \todo{es muss zyklisch sein. Kleinsten invarianten Raum finden}
	eigentlich schon sowieso schon alles 
\end{proof}




Als nächsten Schritt auf dem Weg zum Beweis des Theorems, wird im Folgenden eine Aussage über die Wahl einer Thurston-minimierenden Fläche gezeigt. Solch eine Fläche kann so gewählt werden, dass sie einer Abschätzung genügt, welche genau die Herkunft der Abschätzung in dem Theorem ist. Mit anderen Worten, existiert eine Wahl einer Fläche, die das folgende Lemma sogar mit Gleichheit erfüllt, gilt die Gleichheit bereits im Bezug auf die Alexander Norm.


Da wir uns schon um die endliche Erzeugbarkeit der Homologiegruppen von der Überlagerung bzgl. dem Gruppenring bemüht haben, soll dies nochmal verwendet werden. Da die Homologie nicht endlich erzeugt über den ganzen Zahlen sein muss.


Nur hier wird $b_1=1$ verwendet
    