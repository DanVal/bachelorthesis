%!TEX root = main.tex
\section{Beweis des Theorems}
\label{sec:proofs}
Der Beweis des Theorems und somit der Aufbau dieses Kapitels lässt sich in wenigen Worten grob skizzieren. Zu Beginn werden Konstruktionen entwickelt, die es erlauben konkret mit dualen Flächen zu arbeiten. Anschließend werden zuerst Thurston- dann Alexander-Norm einer Kohomologieklasse $\phi$ mit der ersten Bettischen Zahl ihres Kerns in $\pi_1(G)$ verglichen. Die Transitivität dieser Abschätzung liefert dann das Theorem. 

\subsection{Konstruktionen}
\label{sec:constr}
Bevor die Wahl einer dualen Fläche spezifiziert wird, widmet sich der folgende Teil der zu betrachtenden Überlagerung bezüglich einer Homologieklasse. Will man mit einer Überlagerung Berechnungen anstellen, so sollte man sie möglichst gut kennen. Deswegen folgt nun eine explizite Konstruktion, die sich im späteren Beweis als hilfreich herausstellen wird.

\begin{constr}[Aufschneiden an einer Fläche]
	\label{constr:cut}
	Aus der Überlagerungstheorie ist bekannt, dass zu jeder normalen Untergruppe der Fundamentalgruppe eines hinreichend gut zusammenhängendem Hausdorffraum (insbesondere Mannigfaltigkeiten), auch eine normale Überlagerung existiert, die bis auf Überlagerungsisomorphie eindeutig ist, siehe~\cite[Chapter~1.3]{Hatcher.2002}. Nun definiert aber ein Element $\phi\in H^1(M,\ZZ)$ einen Homomorphismus auf der ersten Homologiegruppe und somit nach den Feststellungen aus Bemerkung~\ref{bem:fundhomologie} auch einen Homomorphismus $\phi = \hat \phi \in \Hom(\pi_1(M),\ZZ)$. 

	Dieses $\phi$ liefert also die normale Untergruppe $[\pi_1(M),\pi_1(M)]\subset \ker \phi \subset \pi_1(M)$, welche wiederrum eine Überlagerung definiert, die fortan $M_\phi$ genannt wird. Zur Berechnung der Thurston-Norm und ihrem angekündigten Vergleich mit $b_1(\ker\phi)=b_1(M_\phi)$ stellt sich die Frage nach einer Abhängigkeit der Überlagerung $M_\phi$ von einer dualen Fläche. Sei also $(S,\partial S) \subset (M\partial M)$ eine eingebettete orientierte Fläche. Da diese Kodimension~$1$ und eine Orientierung hat, ist sie auch zweiseitig, besitzt also eine Umgebung $U \subset M$, sodass ein Diffeomorphismus $S\times (-\epsilon,\epsilon) \to U$ existiert dessen Einschränkung auf $S\times \{0\}$ die Inklusion ist. Das bedeutet, dass die 3-Mannigfaltigkeit an $S$ "`aufgeschnitten"' werden kann (siehe etwa~\cite[Kapitel~4.2]{Burde.2003}), wobei das Aufschneiden bedeutet, das Komplement der Fläche zu betrachten (das Resultat ist offensichtlich eine Mannigfaltigkeit, jedoch können Eigenschaften wie Kompaktheit oder Randbedingungen entfallen). Will man nun die durch Aufschneiden gewonnene Kopien $(M_i)_{i\in \ZZ}, M_i \cong M-S$ wieder verkleben, erweist sich die Zweiseitigkeit der Fläche als günstig sogar notwendig (sonst würde nur \emph{eine} Kopie von $S$ als Rand entstehen). Eine zuvor fixierte zweiseitige glatte Einbettung $h: S\times (-\epsilon,\epsilon) \to M$ liefert nämlich durch $h(S,(-\epsilon,0))$ und $h(S,(0,\epsilon))$ offene Mengen $M_i^-$ und $M_i^+$ in den $M_i$. Durch den strukturerhaltenden Diffeomorphismus $h$ auf sein Bild, können nun $M_i$ und $M_{i+1}$ jeweils entlang $M_i^+$ und $M_{i+1}^-$ verklebt werden --- genauer: $h$ liefert eine Äquivalenzrelation auf der disjunkten Vereinigung 
	\[
		\cdots \sqcup M_{i-1} \sqcup (S \times (-\epsilon,\epsilon)) \sqcup M_i \sqcup  (S \times (-\epsilon,\epsilon)) \sqcup M_{i+1} \sqcup \cdots,
	\]
	sodass der Quotient eine unendlich zyklische Überlagerung mit der offensichtlichen Projektion bildet. Da entlang offenen Mengen verklebt wird, also durch die "`Überlappungen"', ist es möglich den gewonnenen Quotienten mit einer differenzierbaren Struktur zu versehen, so dass die Inklusionen der $M_i$ glatte Einbettungen, durch die Offenheit also Untermannigfaltigkeiten sind.

	Ein weiterer Vorteil dieser expliziten Konstruktion ist es, die Decktransformationsgruppe zu sehen. Sie ist durch einen Erzeuger $t$ über $t\mapsto 1$ zu $\ZZ$ isomorph und unter diesem Isomorphismus entspricht $n\in \ZZ$ einer Translation aller $M_i$ um $n$. Es bleibt nur noch zu zeigen, dass diese Überlagerung auch \textit{die} zu $S$ gehörige Überlagerung ist, die in dem obigen Sinne dem dualen $\phi$ entspricht. Dies sieht man zum Beispiel ein, indem man sich unter dem Isomorphismus $H^1(M;\ZZ)\cong [M,S^1]$ einen glatten Repräsentanten des Bildes von $\phi$ aussucht. Natürlich existiert so einer nach den Bemerkungen in~\ref{sec:poinc} immer, jedoch soll dieser für den gewünschten Nachweis explizit $S$ als Urbild eines regulären Wertes ergeben. Dafür wähle man einen zweiseitigen Kragen $S\times (-\epsilon,\epsilon) \subset M$ und definiere eine Abbildung $f:M\to S^1$, die $S\times 0$ auf $p\in S^1$ abbildet, $M-S \times (-\epsilon,\epsilon)$ konstant auf den antipodalen Punkt von $p$ abbildet und auf $S^1\times (-\epsilon,\epsilon)$ gemäß der Projektion auf den zweiten Faktor fortgesetzt wird. Bezüglich dieser glatt konstruierten Abbildung $f$ ist der Wert $p$ regulär und $f^{-1}p=S$ mit der richtigen Orientierung, falls die Zweiseitigkeit entsprechend dem obigen Kommentar gewählt wurde. Durch paralleles Aufschneiden von $M$ an $S$ und $S^1$ and $p$ (analog wie oben nur 2 Dimensionen tiefer) erhält man folgendes kommutatives Diagramm von Überlagerungen:
	\[
		\begin{xy}
			\xymatrix{M_\phi \ar[r] \ar[d] &\RR \ar[d]\\
						M \ar[r]& S^1}
		\end{xy}
	\]

	Dieses Diagramm ist ist aber nun ein Pullback Diagramm von glatten Faserbündeln. Also ist die Diffeomorphieklasse von $M_\phi$ eindeutig. 
\end{constr}
Wir können festhalten:
\begin{cor}
\label{cor:verklvertr}
		Die unendlich zyklische Überlagerung die durch Aufschneiden und Verkleben an einer zu $\phi$ dualen Fläche entsteht entspricht der normalen Untergruppe $ \ker\phi$ ist.
\end{cor}
\begin{cor}
\label{cor:preimage}
	Jede zu $\phi$ duale Fläche kann als Urbild eines regulären Wertes einer glatten Abbildung $M\to S^1$ dargestellt werden, mit $H_1(M \to S^1)=\phi$
\end{cor}

\begin{constr}[Graph einer orientierten 1-kodimensionalen Untermannigfaltigkeit]
	\label{constr:graph}
	Sei $S$ ein beliebiger zu $\phi \in H^1(M,\ZZ)$ eigentlich eingebetteter, orientierter Repräsentant. Wir haben gesehen, dass mit obiger Konstruktion eine Abbildung $f:M\to S^1$ entsteht mit $f^{-1}p=S$. Diese Abbildung soll in dieser Konstruktion faktorisiert werden über einen Graphen.
	Bezeichne $S=S_1\sqcup \cdots \sqcup S_n$ und $M-S = M_1 \sqcup \cdots \sqcup M_m$ die Zusammenhangskomponenten von $S$ bzw. $M-S$. Betrachte nun den gerichteten Graphen $G$ dessen Knoten bijektiv den Komponenten $M_i$ entsprechen und dessen Kanten aus den Komponenten $S_i$ mit ihrer Orientierung hervorgehen, also ein Graph mit $m$ Knoten und $n$ Kanten, wobei eine Kante von einem Knoten zu einem anderen verläuft, wenn ihre assoziierten Komponenten $M_i, M_j$ durch das entsprechende Flächenstück von $S$ getrennt werden, sodass die Komponenten das zweiseitige Flächenstück an der negativen beziehungsweise positiven Seite berühren je nachdem ob die Kante vom assoziierten Knoten aus oder eingeht.

	Mit genau diesen zweiseitigen Umgebungen der Flächenkomponenten ist es möglich sich ähnlich wie oben eine Abbildung $M \to G$ zu definieren, welche die Assoziierungen respektiert. Dafür betrachte man die glatte Einbettung (welche die Zweiseitigkeit der Komponenten respektiert):
	\[
		\sqcup (S_i \times (-\epsilon,\epsilon)) \stackrel = \longrightarrow S \times (-\epsilon,\epsilon) \into M
	\]
	Dann existiert analog zur obigen Konstruktion die Quotientenabbildung $q:M\to G$ auf den Graph, durch Kollabieren der $M_i\cap (M -S \times(-\epsilon,\epsilon))$ auf ihre Knoten und Projektion von $(-\epsilon,\epsilon)$ auf das Innere der Kanten des Graphen. Betrachtet man außerdem $G \to S^1$ die Abbildung die jede Kante entsprechend ihrer Richtung einmal um die Sphäre $S^1 = I/\partial I$ abbildet und die Knoten auf den antipodalen Punkt des ausgezeichneten Punktes $p\in S^1 = [\partial I]$. Bezüglich der Komposition der beiden Abbildungen $M \to S^1$, ist nun $\phi$ das Bild des Erzeugers von $H^1(S^1)$ unter der Rückziehung auf der Kohomologie, da $M \to S^1$ eine zu $S$ duale Kohomologieklasse definiert (da $S=(M\to S^1)^{-1}p$ ist). Sei $f$ nach wie vor die Abbildung aus der letzten Konstruktion so erhalten wir:
	
	\begin{align}
		\begin{xy}
				\xymatrix{M \ar[r] \ar@/^1pc/[rr]^f & G \ar[r] & S^1 }
			\end{xy}
		\label{eq:graphlift}
	\end{align}
	Da $M$ zusammenhängend ist, ist es auch $G$.
\end{constr}

Im Folgenden wollen wir häufig den Spezialfall betrachten, dass $\phi: H_1(M) \to \ZZ$ surjektiv ist. 
\begin{defn}
	Ein solches $\phi \in H^1(M;\ZZ)$ heißt primitiv.
\end{defn}

\subsection{$\thur \cdot$-minimierende Flächen}
    
Nun wollen wir uns eine besondere Art von Flächen anschauen, nämlich die Repräsentanten der zu $\phi$ dualen Klasse, bei denen $\chi_-$ minimal ist. Dass immer ein Repräsentat existiert, der gewissen Eigenschaften genügt, sichert der folgende Satz:

\begin{lem}
	\label{lem:minS}
	Sei $\phi \in \Hom (\pi_1(M),\ZZ)$ ein primitives Element mit $b_1(\ker\phi)<\infty$. Dann existiert eine zusammenhängende Thurstonnorm-minimierende Fläche $(S,\partial S) \subset (M,\partial M)$ mit $\phi \mapsto [S]$ unter Poincaré Dualität und $b_2(S)=b_3(M)$, so dass folgende Abschätzung erfüllt ist:
	\[
	b_1(S) \leq b_1(\ker(\phi))
	\]
\end{lem}
\begin{proof}
	Wähle unter allen Thurston-Norm-minimierenden Flächen eine orientierte Fläche $S$ mit einer geringsten Anzahl an Zusammenhangskomponenten.\\
	\textit{Behauptung: Diese Fläche ist zusammenhängend}.
	Man betrachte den aus $S$ entstehenden Graphen nach Konstruktion~\ref{constr:graph}.
	Wir werden zeigen, dass unter der Minimalitätsanforderung von $S$ unter diesen Identifikationen klar ist, dass $G$ ein topologischer Kreis ist und $G\to S^1$ die Identität. Das impliziert die Behauptung.

	Dafür betrachte man folgendes Diagramm von Pullbacks von Überlagerungen:

	\[
	 	\begin{xy}
	 		\xymatrix{
	 			M_\phi \ar[r] \ar[d] & G_\phi \ar[d] \ar[r] & \RR \ar[d]\\
	 			M \ar[r] & G \ar[r] &S^1
	 		}
	 	\end{xy}
	 \] 
	 Man erhält unendlich zyklische Überlagerungen, also unendlich zyklische Decktransformationsgruppen. Jedes Element in $\pi_1(G_\phi)$ ist homotop zu einem Lift einer Schleife  aus $\pi_1(G)$ (Vernachlässigung des Basispunktes ist in dieser Situation keine Einschränkung). Unter der oben gezeigten Kompatibilitätsvorraussetzung dieser Überlagerungen durch Pullbacks, mit den Überlagerungen durch Aufschneiden an dualen Flächen (deswegen die suggestive Schreibweise $M_\phi$), entsteht $G_\phi$ durch "`Aufschneiden an den Knoten"', also liftet jede Schleife aus $G$ trivial. Folglich ist $G_\phi$ einfach zusammenhängend, überlagert $G$ also universell. Somit ist $\pi_1(G)= \ZZ$. Also ist $G$ als Graph vom Homototyp ein Kreis. Da $G$ aber auch die Kompaktheit von $M$ erbt, ist nur noch die Existenz von Knoten ohne ein- oder ausgehende Kanten auszuschließen, um zu zeigen das $G$ ein topologischer Kreis ist. Diese ist aber durch die Minimalitätseigenschaft im Bezug auf die Komponenten der gewählten Fläche ausgeschlossen, da solche Kanten einer nullhomologen Kette entsprechen. Dies sieht man, falls $M$ geschlossen ist wie folgt ein: Sei $M_i$ ein Knoten der nur eingehende oder nur ausgehende Kanten besitzt, etwa $S_i, i \in I$. Dann ist ohne Beschränkung der Allgemeinheit $(M_i,\sqcup_{i\in I}S_i)$ eine Mannigfaltigkeit mit Rand (je nach Geschmack betrachte man sonst die umgekehrte Orientierung auf den $S_i$ oder das Komplement von $M_i$). Die Homologieklasse von $\hat S = \sqcup_{i \in I} S$ ist das Bild der Fundamentalklassen unter der von der Inklusion $H_2(i:\hat S \into M)$ induzierten Abbildung. Doch $ \hat S \stackrel i \into M = \hat S \stackrel j \into M_i \cup \hat S \stackrel k \into M$ faktorisiert und mit Funktorialität faktorisiert auch $H_2(i)=H_2(k)H_2(j)$ aber $H_2(j)=0$. Hier wurde benutzt, dass der Rand einer Mannigfaltigkeit als Klasse mit Kodimension $1$ nullhomolog ist (aus der langen exakten Sequenz für das Paar folgt, dass die Inklusion von Rändern auf der Homologie der Kodimension 1 eine triviale Abbildung induziert, da der Randoperator einen Isomorphismus induziert). Falls $M$ jedoch nicht geschlossen ist, so folgt mit einem einfachen Schnittzahlenargument, dass jede Homologieklasse als transversale Schleife ausgewertet auf $[\hat S]$ (als duale Kohomologieklasse), die Summe der transversalen Schnitte mit Vorzeichen ergibt, also $\hat S$ dual zu dem trivialen Homomorphismus $\pi_1(M) \stackrel 0\to \ZZ$ ist, also selbst nullhomolog. Folglich würde dies eine Fläche mit $|I|$ weniger Komponenten und somit einen Widerspruch liefern:
	 \[
	 	[S]=[\sqcup_{i\not \in I} S_i\bigsqcup \sqcup_{i\in I} S_i] = [\sqcup_{i\not \in I} S_i ]+[ \sqcup_{i\in I} S_i] = [\sqcup_{i\not \in I}S_i]
	 \]
	 Betrachtet man nun, wissend dass $G$ ein topologischer Kreis ist, die induzierte Abbildung auf der Homologie $(G\to S^1)_*$, so ist diese ein Isomorphismus, da $\phi$ primitiv ist. Also besitzt $G$ nur eine Kante und die Fläche $S$ ist zusammenhängend.

	 Die nächste Gleichheit --- \emph{die Übereinstimmung des Ranges auf den Top-Homologien von $S$ und $M$} --- hängt von der Existenz eines Randes ab. Da $(S,\partial S) \subset (M,\partial M)$ folgt aus $\partial S \neq \emptyset$ direkt $b_2(S)=b_3(M)=0$. Falls $S$ aber leeren Rand hat, gilt $b_2(S)=1$ und es muss $b_3(M)=1$ gezeigt werden. Dazu wird die Existenz eines Randes von $M$ widerlegt:

	 Nach Annahme existeren nur Tori als Randkomponenten. Sei $T \subset \partial M$ eine solche Randkomponente. Da $S$ keinen Rand hat, also $T$ nicht berührt, enthält die durch Aufschneiden an $S$ erhaltene unendlich zyklische Überlagerung $M_\phi$ auch unendlich zyklisch viele Kopien von $T$ als Randkomponenten. Genauer: unter Verwendung dieser Aufschneide-Konstruktion~\ref{constr:cut}, liftet $T$ in jedes $M_i \cong M - S \supset T$, auf der sich die Projektion zu einem Diffeomorphismus einschränkt. Im Folgenden soll die Notation $\overline M_i$ für die (wieder) kompakte (Unter-)Mannigfaltigkeit von $M_\phi$ verwendet werden, die den Fundamentalbereich darstellt, der $M_i$ enthält. Zusammen mit der langen exakten Sequenz für eine kompakte orienierbare Mannigfaltigkeit $(N,\partial N)$
	\[
	 \begin{xy}
	 	\xymatrix{
	 	H_2(N;\QQ) \ar[r]&  H_2(N,\partial N ; \QQ) \ar[r]^\delta & H_1(\partial N;\QQ) \ar[r]^{i_*}& H_1(N;\QQ) \\
	 	& &H^1(N;\QQ) \ar[ru]_{\cong} \ar[lu]^{\text{Lefschetz Dualität: }\cong \qquad~}&}
	 \end{xy}
	 \] 
	 und der daraus folgenden Abschätzung
	 \[
	 	b_1(\partial N) = \dim(\im \delta) + \dim(\im i_*) \leq  2b_1(N)
	 \]
	 erhält man für jede kompakte zusammenhängende (man bemerke die implizite Anforderung an $I\subset \ZZ$) Untermannigfaltigkeit der Form 
	 \[
	  	\bigcup_{i\in I} \overline M_i \subset M_\phi 
	  \]
	  die Abschätzung:
	  \[
	   	b_1(\bigcup_{i\in I} \overline M_i)\geq \frac{1}{2}b_1(\partial \bigcup_{i\in I} \overline M_i) \geq \frac{1}{2}b_1(\sqcup_{i \in I}T) = |I|
	  \]


	 \noindent Nun folgt aber aus der Mayer Vietoris Sequenz (für entsprechende offene Umgebungen\footnote{Stichworte hier: das allgemeine Theorem über Tubenumgebungen, Umgebungsdeformationsretrakt}) die exakte Sequenz:
	  \[
	  	\cdots\to H_1(S\sqcup S;\QQ) \to H_1(\bigcup_{i \in I} \overline M_i;\QQ) \oplus H_1(M_\phi -\bigcup_{i \in I} \overline M_i;\QQ) \to H_1(M_\phi;\QQ) \to 0
	  \]
	  und somit
	  \[
	  	b_1(M_\phi)= b_1(\bigcup_{i \in I} \overline M_i)+b_1(M_\phi -\bigcup_{i \in I} \overline M_i)-b_1(S\sqcup S) \geq |I| -2b_1(S)
	  \]
	  Da aber $b_1(M_\phi)$ nach Voraussetzung endlich ist und $|I|$ beliebig groß, folgt also dass der Rand von $M_\phi$ keine Tori enthält und somit leer ist.

	  Um nun noch\emph{ die Abschätzung $b_1(M_\phi) \leq b_1(S)$ }zu zeigen, wird erneut die Konstruktion der Überlagerung durch Aufschneiden und Verkleben zur Hilfe genommen. Da $\ker\phi \tensor \QQ \cong H_1(M_\phi;\QQ)$ nach Voraussetzung ein endlich erzeugter $\QQ$-Vektorraum ist, wird $H_1(M_\phi;\QQ)$ von einem kompakten Teilraum, etwa der Untermannigfaltigkeit $\overline M_1 \cup \cdots \cup \overline M_k \into M_\phi$ und somit auch $t^k(\overline M_1 \cup \cdots \cup \overline M_k )  = \overline M_{k+1} \cup \cdots \cup \overline M_{2k}\into M_\phi$, erzeugt (heißt: die Inklusionen erzeugen Epimorphismen auf der ersten Homologie). Mit diesem Wissen liefert die folgende exakte Sequenz (induziert nach Mayer-Vietoris) die gesuchte Abschätzung:
	  \[
	  	\cdots \to H_1(S;\QQ) \to H_1(\bigcup_{i\leq k}\overline M_i;\QQ) \oplus H_1(\bigcup_{i>k} \overline M_i;\QQ) \onto H_1(M_\phi;\QQ)
	  \]
\end{proof}

\begin{bem}
Da die erste Homologie der Überlagerung natürlich bessere Chancen hat, als Modul, durch Vergrößerung des Grundrings, über dem Gruppenring $\ZZ\subset \laurent \ZZ t$ endlich erzeugt zu sein, stellt sich die Frage, ob, wie und warum es sinnvoll oder möglich wäre das eben bewiesene für diesen Fall zu verallgemeinern. Dies wird später mit Hilfe der weiteren Lemmas in Kapitel~\ref{verallggruppenring} diskutiert. 
\end{bem}
\begin{bem}
	Tatsächlich liefert dieses Lemma sogar \emph{die} Abschätzung des Theorems~\ref{thm:haupttheorem} wie wir nachher sehen werden. Gilt hier also Gleichheit, so folgt auch die Gleichheit in dem Theorem. Diese Bemerkung bietet also dem Leser die Möglichkeit noch einmal kurz inne zu halten und sich die Natur der Abschätzung anhand der vorhergehenden Seiten zu verdeutlichen.
\end{bem}

\subsection{Darstellungen des Alexander Ideals}
    
Nun vergleicht das vorangegangene Kapitel also die Thurston Norm einer Kohomologieklasse mit dem Rang ihres Kerns. Wie letzerer mit der Alexander Norm in Verbindung steht, stellt folgendes Lemma (vgl.~\cite[Assertion~4]{Milnor.2009}) für $b_1(M)=1$ fest:
\begin{lem}
	\label{lem:charPol}
	Es sei wieder $\phi \in H^1(M;\ZZ)$ eine primitive Klasse und $\ker \phi \tensor \QQ$ ein endlich dimensionaler Vektorraum. Weiter sei $t$ ein Erzeuger der Decktransformationsgruppe von $M_\phi$, sodass analog zu der $\laurent \ZZ t$-Modul Struktur auf dem $\ZZ$-Modul $H_1(M_\phi;\ZZ)$ der $\QQ$-Modul $H_1(M_\phi;\QQ)$ als Modul über dem rationalen Gruppenring $\QQ[\langle t \rangle]$ aufgefasst werden kann, wobei der Gruppenring kanonisch mit $\QQ[t^{\pm 1}]$ identifiziert wird. Dann ist das Elementarideal $E_0(H_1(M_\phi))\subset \QQ[t^{\pm1}]$ bezüglich einer $\QQ[t^{\pm 1}]$-Präsentation ein Hauptideal das von dem charakteristischen Polynom des induzierten Automorphismus $t_*$ erzeugt wird.
\end{lem}
\begin{cor}
	Insbesondere erzeugt für $b_1(M)=1$ das Alexander Ideal ein Hauptideal. \qed
\end{cor}
Um zunächst die Aussage klarzustellen, sei darauf hingewiesen, dass nach dem universellen Koeffiziententheorem für Homologie gilt, dass $H_1(M_\phi;\QQ) \cong H_1(M_\phi;\ZZ)\tensor_\ZZ \QQ$ und weiter folgt das entsprechende Corollar für das Alexander Polynom durch Proposition~\ref{prop:tensoring}.
In dem Beweis wollen wir nutzen, dass $\laurent\QQ t$ ein Hauptidealring ist. Deswegen folgendes Hilfsmittel:
\begin{lem}
\label{lem:QThauptidealring}
	$\laurent\QQ t$ ist ein Hauptidealring.
\end{lem}
Natürlich könnte man $\QQ$ für dieses Lemma durch jeden beliebigen Körper $\KK$ ersetzen. Allerdings entfällt durch Tensorieren mit einem endlichen Körper nicht jegliche Torsion, entsprechend birgt das mögliche Probleme bei Anwendung des universellen Koeffiziententheorems, da der $\Tor$ Anteil nicht zwangsweise verschwindet und somit hält Proposition~\ref{prop:tensoring} nicht mehr.
\begin{proof}
	Da $\QQ$ ein Körper ist, ist $\QQ[t]$ ein Hauptidealring. Nun ist allgemeiner die Eigenschaft für einen Ring ein Hauptidealring zu sein abgeschlossen unter Lokalisierung. Der Beweis läuft durch Betrachtung der kanonische Lokalisierungsabbildung:
	\[
		\alpha : \QQ[t] \to \laurent \QQ t
	\]
	Die Ideale in der Lokalisierung sind genau die erweiterten Ideale aus dem ursprünglichen Ring, wegen der Erhaltung durch $\laurent \QQ t \supset I=\alpha_*(\alpha^*(I))$, wobei $\alpha^* = \alpha^{-1}$ die Kontraktion eines Ideals bezeichnet und $\alpha_*(I) = \laurent \QQ t \cdot \alpha(I)$ die Erweiterung eines Ideals. Da aber jedes Element in $\laurent \QQ t$ durch Multiplizieren mit der Einheit $t^N$ im Erzeugnis eines Polynoms liegt, wird ein Erzeuger aus dem Hauptideal $\alpha^*(I)$ auf einen Haupterzeuger in $I$ abgebildet. 
\end{proof}
 In der Tat ist dieser Beweis in dem Fall des Laurentrings doch mehr als einleuchtend, allerdings merkt man sich so auch gut den Beweis für allgemeine Lokalisierungen, indem man bemerkt, dass sich das Multiplizieren mit der Einheit $t^N$ einfach erreichen lässt durch Multiplikation mit Elementen aus der multiplikativen Menge an der lokalisiert wurde --- den neuen Einheiten aus dem lokalisierten Ring.

\begin{cor}
	Die ganzzahligen Alexander Polynome $\Delta_\phi^i$ sind durch $A_\phi(G) \tensor \QQ$ vollständig charakterisiert.
\end{cor}
\begin{proof}
	Der Beweis folgt aus Zusammentragen der Ergebnisse aus Beispiel~\ref{bsp:hauptidealelementarteiler}, Proposition~\ref{prop:tensoring} und Lemma~\ref{lem:QThauptidealring}.
\end{proof}

 Nach dem Einschub algebraischer Natur, nun der Beweis des Lemmas~\ref{lem:charPol}:
\begin{proof}[Beweis von Lemma~\ref{lem:charPol}]
	Da $H_1(M_\phi,\QQ)$ ein endlich dimensionaler Vektorraum ist, werden durch den Erzeuger $t$ des Quotienten $\pi_1(M)/\ker(\phi) \cong \ZZ$ der Decktransformationen Relationen auf den Basiselementen $x_1,\cdots,x_n$ eingeführt:	
	\begin{align*}
		t_*x_1 &= \sum a_i^1 x_i \\
				&\vdots \\
		t_*x_n &= \sum a_i^n x_i
	\end{align*}
	Diese Gleichungen definieren genau die quadratische Matrix $A$ des Automorphismus von Vektorräumen $t_* \in \Aut (H_1(M_\phi;\QQ))$ bezüglich der Basis $(x_i)_i$, die also als Spalten die $a^i$ hat. Durch subtrahieren der obigen Gleichungen, erhält man eine formale Matrix der Form $A-tI$. Diese Matrix ist aber gleichzeitig die Präsentationsmatrix der freien Auflösung:
	\[
		\begin{xy}
			\xymatrix@L+5pt{\laurent \QQ t ^n \ar[r]_{e_r\mapsto \sum a_i^rx_i-tx_r} & \laurent \QQ t ^n \ar[r] & H_1(M_\phi;\QQ) \ar[r] &0\\}
		\end{xy}
	\]
	Entsprechend ist die Determinante dieser Matrix das Elementarideal bezüglich $\laurent \QQ t$, $E_0(M_\phi)=det(A-tI)=\chi(A) \subset \laurent \QQ t $. 
\end{proof}



Diese Überlegungen liefern nun einen Zusammenhang zwischen den Thurston-Norm-minimierenden Flächen, deren erste Bettizahl nach Lemma~\ref{lem:minS} immer mit oberer Schranke $b_1(\ker\phi)$ gewählt werden kann, und dem Grad des Alexander Polynoms von $\phi$. Weiter, falls also $b_1(M)=1$, so folgt bereits der Zusammenhang mit der Alexander-Norm von $M$. Dies wollen wir in folgendem Corollar festhalten.
\begin{cor}
\label{cor:degreealex}
	Sei $\phi \in H^1(M;\ZZ)$ eine primitive Klasse. Dann gilt:
	\[
		\dim(\ker\phi \tensor \QQ) = \Grad(\Delta_\phi) 
	\]
\end{cor}
\begin{proof}
	Nach dem vorhergehenden Corollar gilt $\dim \Delta_\phi = n+1 = \Grad(\chi(t_*)) = \Grad(\Delta_\phi)$, wobei $t$ Erzeuger der unendlich zyklischen Decktransformationsgruppe ist. 
\end{proof}

	Für den nächsten Beweis ist es nützlich von Homologie mit getwisteten Koeffizienten zu sprechen: Sei $G=\pi_1(M)$, dann wird $\ZZ[ab(G)]$ durch Linksmultiplikation mit Elementen aus $G$ zu einem $\ZZ[G]$-Linksmodul. Ebenso wird die zelluläre Kettengruppe $C_i(\hat M)$ der universellen Überlagerung $\hat M$, mit den induzierten Automorphismen der Decktransformationsgruppe kanonisch identifiziert mit $G$ zu einem $\ZZ[G]$-Rechtsmodul, wobei es hier natürlich wichtig ist, dass die zelluläre Struktur auf $\hat M$ von $M$ vererbt ist, also die Zellen genau den Zusammenhangskomponenten der Urbilder von Zellen in $M$ entsprechen --- nur so erhält man eine freie Basis aus den Zellen von $M$. Somit erhält man zu einem gegebenen zellulären Kettenkomplex $C_3(\hat M,\hat p) \to C_2(\hat M,\hat p) \to C_1(\hat M,\hat p) \to C_0(\hat M,\hat p)$, wobei $\hat p = \pi^{-1}(p)$ das Urbild einer Nullzelle $p\in M$ ist, der universellen Überlagerung den tensorierten Kettenkomplex 
\begin{align}
			C_3(\hat M,\hat p)\tensor_{\ZZ[G]}\ZZ[ab(G)] \to C_2(\hat M,\hat p)\tensor_{\ZZ[G]}\ZZ[ab(G)]& \to C_1(\hat M,\hat p)\tensor_{\ZZ[G]}\ZZ[ab(G)] \label{eq:twistedcomplex}\\
			& \to C_0(\hat M,\hat p)\tensor_{\ZZ[G]}\ZZ[ab(G)]  \notag
	\end{align}	
	Man beachte hierbei, dass es sich nun um $\ZZ$-Moduln handelt, da der zugrundeliegende Gruppenring nicht kommutativ sein muss. Aber aus offensichtlichen Gründen, handelt es sich um einen Kettenkomplex von $\ZZ[ab(G)]$-Moduln. Bezeichnet man den Kettenkomplex~\eqref{eq:twistedcomplex} mit $C_\text{\textbullet}(M;\ZZ[ab(G)])$, so können wir nun definieren:

	\begin{defn}
		Definiere die Homologie mit getwisteten Koeffizienten von $M$ als
		\[
		 H_i(M,p;\ZZ[ab(G)])=H_i(C_\text{\textbullet}(M;\ZZ[ab(G)])) 	
		 \] 
	\end{defn}

	Man sieht leicht ein, dass dies wohldefiniert ist und nicht von der Zellzerlegung von $M$ abhängt. Weiter ergibt sich, dass $H_1( M_{ab(G)} ,  p_{ab(G)}) \cong H_1(M,p;\ZZ[ab(G)])$ natürlich isomorph sind. Der Beweis ergibt sich direkt aus dem Resultät aus der Überlagerungstheorie, dass die universelle Überlagerung buchstäblich universell überlagert, also insbesondere $M_{ab(G)}$, siehe etwa~\cite[Kapitel~1.3]{Hatcher.2002}.
    
    Für die Abschätzung der beiden Halbnormen, ist es hier zielführend, dass das Alexander Ideal einer 3-Mannigfaltigkeit eine nicht allzu komplizierte Gestalt annehmen kann. Folgendes Theorem (aus McMullen~\cite{MCMULLEN.2002}) sogar, dass $I_M(G)$ Definition, siehe Kapitel~des Alexander Ideals ein Produkt maximal dreier Faktoren ist, von denen eines immer der größte Teiler --- das Alexander Polynom --- ist. 

\begin{thm}
\label{thm:keralexnorm}
	Sei $G$ die Fundamentalgruppe einer 3-Mannigfaltigkeit $M$, die den Voraussetzungen des Haupttheorems genügt und $\phi:G \to H_1(G)/T \cong ab(G) \cong \ZZ^{b_1(G)}$ die Quotientenabbildung auf den maximalen frei abelschen Quotienten. Dann gilt:
	\[
		I_M(G)=E_1(m(G)/m(\ker\phi)m(G)) = m(ab(G))^{1+b_3(M)}\cdot (\Delta) 
	\]
\end{thm}
\begin{proof}
	Wie obige Bemerkung gestattet, zielt der Beweis darauf ab eine Präsentation der getwisteten Homologie $H_1(M,p;\ZZ[ab(G)]\cong I_m(G)$ zu finden. Dies soll explizit konstruiert werden, durch Betrachtung einer konkreten CW-Struktur und zellulärer Homologie. Glatte 3-Mannigfaltigkeiten sind stets triangulierbar, entweder überzeugt man sich durch die grundsätzliche Triangulierbarkeit von 3-Mannigfaltigkeiten nach dem Satz von Moise oder man glaubt an die Triangulierbarkeit von glatten Mannigfaltigkeiten, die etwa nach dem Satz von Whitehead folgt oder durch die Einbettbarkeit jeder Mannigfaltigkeit in den $\RR^N$, die von Whitney gezeigt wurde. Jedenfalls sei $\tau$ eine Triangulierung von $M$. Um eine explizite und vor allem möglichst simple CW-Struktur zu erhalten, wählt man sich nun zwei möglichst "`große"' Bäume $B,B'$ in den Strukturen von $M$, die der Vereinfachung der CW-Struktur dienen sollen und den Homotopietyp von $M$ nicht ändern. Und zwar sei $B$ ein maximaler Baum im Graph des 1-Skeletts von $\tau$, also ein zusammenhängender, kreisloser Graph. Für $B'$ betrachtet man folgende Konstruktion eines Graphen $G$:  man nehme die Baryzentren aller $3$-Simplices als Knotenmenge und die Kanten verbinden diese Knoten gemäß der trennenden $2$-Simplices von $\tau$. Offensichtlich gilt für jeden Baum in $G$, dass die zugehörige Vereinigung von $3$-Simplices die aus seinen Knoten hervorgeht ein topologischer Ball ist und der Randoperator auf der Summe der zugehörigen 3-Simplices im Kettenkomplex nur 2-Simplices des Randes ergibt (die inneren 2-Simplices treten jeweils als Differenz auf). Sei also $B'$ ein maximaler Baum in $G$. Das bedeutet insbesondere, dass $G/B'$ diffeomorph zu einem Bouquet von Kreisen ist, denn $G$ ist zusammenhängend, da $M$ zusammenhängend ist. Da Homologie eine Homotopieinvariante ist, betrachte man den zellulären Kettenkomplex der aus dem CW-Komplex $M/B$ hervor geht, mit gewählter Struktur, dass man durch die Konstruktion von $B'$ eine einzige 3-Zelle erhält, die aus Entnahme aller 2-Simplices entsteht die zu den Kanten aus $B'$ gehören. Die Projektion $X \to X/A$ ist stets eine Homotopieäquivalenz, falls $A\into X$ eine Kofaserung und $A$ zusammenziehbar ist siehe Whitehead Kap 1 Corollar 5.13 und $(M,B)$ ist als CW-Paar eine Kofaserung. Aus dieser Struktur geht folgender zellulärer Kettenkomplex von $\ZZ[ab(G)]$-Moduln $C_i=C_i^{zell}(\widehat{M/B},\hat p)\tensor_{\ZZ[G]}\ZZ[ab(G)]$ hervor:
	\[
		\begin{xy}
		\xymatrix{C_3 \ar[r]^{\partial_3}  &C_2 \ar[r]^{\partial_2}  &C_1 \ar[r]^{\partial_2} &C_0 }
		\end{xy}
	\]
	Sei $M$ nun zunächst geschlossen. Dann folgt mit Eulercharakteristik und Poincaré Dualität (und Homotopieinvarianz beziehungsweise Wohldefiniertheit der genutzten Invarianten):
	\[
		0=b_0(M)-b_3(M)+b_2(M)-b_1(M) = \chi(M) 
	\]
	\[
		= \#\{\text{Differenz der Zellen in gerader und ungerader Dimension}\}
	\]
	also folgt dass die gewählte CW-Struktur jeweils $n$ Zellen in Dimension 1 und 2 hat. Betrachtet man nun die relative Homologie zu der ausgezeichneten $0$-Zelle $p$, folgt direkt für den Alexander-Modul $A_M(G)= H_1(M,p;\ZZ[ab(G)])=C_1/\im(\partial _2)$? Man beachte, dass die Moduln in dem betrachteten Kettenkomplex frei sind, denn: 
	\begin{align*}
		C_i &=C_i^{zell}(\widehat{M/B},\hat p)\tensor_{\ZZ[G]}\ZZ[ab(G)] = \ZZ[G]^n \tensor_{\ZZ[G]} \ZZ[ab(G)] = \ZZ[ab(G)]^n\\
		C_i^{zell}(\widehat{M/B}) &= \bigoplus_{g \in G} C_i^{zell}(M/B) = \ZZ[G]^n
	\end{align*}
	Also erhält man eine Präsentation von diesem Modul durch:
	\[
		C_2 \stackrel {\partial_2} \longrightarrow C_1 \onto C_1/\im(\partial_2) = A_M(G)
	\]
	$\partial_2$ lässt sich für gewählte $\ZZ[ab(G)]$-Basen von $C_1$ und $C_2$ als Matrix darstellen, dessen $(n-1)\times (n-1)$ Minoren die Erzeuger von $I_M(G)$ liefern. Für die Berechnung dieser Matrix, berechnen wir zuerst $\partial_1$ und $\partial_3$.

	Man fixiere für eine Basis einen Fundamentalbereich genauer einen Basispunkt $\hat e_0 \in \hat p = \pi^{-1}(p)$ aus dem Urbild der 0-Zelle, an diesem kann eine $\ZZ[G]$ Basis der $C_i^{zell}(\widehat{ M/B})$ gewählt werden. Die Vereinigung dieser gewählten Zellen liegt dann dicht in dem Fundamentalbereich. Ist also eine beliebige Zelle $e \in C_i^{zell}(\widehat{ M/B})$ gegeben, so entsteht diese aus einem Element der fixierten Basis aus dem fixierten Fundamentalbereich durch Multiplikation mit einem $g\in G$ (genauer bedeutet dies eine Anwendung der Decktransformation, die Zellen auf Zellen abbildet), $g\hat e = e$. Um also einzusehen, wie sich eine Basis für die freien $\ZZ[ab(G)]$ Moduln unter den obigen Isomorphismen verhält, stellt man fest, dass für ein beliebiges $e\in C_i^{zell}(\widehat{ M/B})$ gilt: $e\tensor 1 \mapsto g\hat e \tensor 1 = \hat e \tensor [g] \mapsto [g]\hat e $, wobei also die fixierten Zellen die kanonischen Koordinaten von $\ZZ[G]$ und $\ZZ[ab(G)]^n$ darstellen. Dies lässt nun zu, dass man die komplizierte Situation fast vergessen könnte, indem man die kanonischen Koordinaten verwendet. 

	Was geschieht nun durch Anwendung von $\partial_1$? Sei $e$ eine generische 1-Zelle in $C_1^{zell}(\widehat{ M/B})$, dann ist $e=g\hat e$, wobei $\hat e$ aus der fixierten Basis stammt, und $\overline{\pi(\hat e)}$ stellt per Definition das Bild einer (nicht notwendigerweise glatten, es sind auch keine differenzierbaren Eigenschaften von $M/B$ gefordert!) Abbildung $S^1 \to M/B$ dar, also ein Element $\hat g \in \pi_1(M/B,p) = G$. Also berechnet sich das Bild von $\hat e$ unter der Randabbildung durch den (natürlich nicht geschlossenen) Lift dieses Elementes, nämlich $\partial(e\tensor 1)=\partial(g \hat e \tensor 1) = \partial(\hat e \tensor [g])= (\hat e_0 - \hat g \hat e_0)\tensor [g]= (1-\hat g)g\hat e_0 $. Die zu Beginn recht kompliziert wirkende Situation, wurde in den letzten Abschnitten und durch allerlei Verträglichkeiten deutlich gelockert und ermöglicht es nun ohne viel Verwechslungsgefahr zwischen verschiedenen Identifikationen zu wechseln. Sei also $\hat e_1^1 ,\cdots, \hat e_1^n$ die fixierten 1-Zellen in der Überlagerung, also eine Basis des $\ZZ[F]^n$ dann erhalten wir die Matrixdarstellung des Homomorphismus $\ZZ[F]^n \stackrel {\partial_1} \longrightarrow \ZZ[F]=\langle \hat e_0\rangle$
	\[
		\partial_1 = (1-\hat g_1, \cdots , 1- \hat g_n) 
	\]
	wobei $\hat g_i \in \pi_1(M/B,e_0)=G$ die Klasse von $\hat e_1^i$ unter der Projektion in $M/B$ ist.

	Ähnlich fährt man nun mit der Berechnung von $\partial_3$ fort. Sei $\hat e$ eine der fixierten 2-Zellen aus $C_2^{zell}(\widehat{ M/B})$, also ein Element der $\ZZ[G]$-Basis von $C_2^{zell}(\widehat M/B)$ oder anders: ein Element der $\ZZ[F]$-Basis von $C_2$. Erinnert man sich an die Wahl von $\tau$, so ist $\hat e$ in $M$ Seite von zwei 3-Simplices. In $B'$ existiert nun ein eindeutiger (Geodätische in Bäumen sind eindeutig) Pfad, der die Baryzentren dieser zwei 3-Simplices verbindet. Schließt man diesen Pfad nun zu einer Schleife, durch die (nach obiger Konstruktion von $G\supset B'$) zu $\hat e$ gehörige Kante, so liefert diese Schleife wieder ein Element $\hat h \in \pi_1(M/B)=G$. Um nun das Bild von der fixierten 3-Zelle $\hat e_3$ unter $\partial_3$ festzustellen, fasse man $\tau/B$ als CW-Struktur auf und erwäge man folgendes Diagramm bezüglich der fixierten 2-Simplex Basis $\hat e_2^i$ und dazugehörigen $\hat h_i \in G$:
	
\begin{center}

\begin{tikzpicture}
	\node (P0) at (0,0) {$C_3^{\tau,zell}(\widehat {M/B}) $};
	\node (P1) at (-3,-3) {$C_3^{zell}(\widehat {M/B})$} ;
	\node (P2) at (3,-3) {$C_2^{zell}(\widehat {M/B})$};
	\node [font=\scriptsize] (P0r) at (1.7,0.3) {$\sum \hat\sigma_3$};
	\node [font=\scriptsize] (P0l) at (-1.7,0.3) {$\sum \hat\sigma_3$};
	\node [font=\scriptsize] (P11) at (-4.7,-2.7) {$\hat e_3$} ;
	\node [font=\scriptsize] (P22) at (4.7,-2.7) {~$\sum (1-\hat h) \hat e_2^i $};
	\draw
	(P0) edge[->,>=angle 90] node[left,outer sep=3pt] {$\Sigma$} (P1)
	(P1) edge[->,>=angle 90]  (P2)
	(P0) edge[->,>=angle 90] node[right,outer sep=3pt] {$\partial$} (P2)
	(P0r) edge[|->] (P22)
	(P0l) edge[|->] (P11);
	\end{tikzpicture}
\end{center}


	Also stellt sich heraus, dass für die fixierten Zellen $\hat e_2^1,\cdots, \hat e_2^n$ als Identifikation einer Basis von $\ZZ[ab(G)]$, sich die Abbildung $\partial_3$ als Matrix auffassen lässt:
	\[
		\partial_3 = (1-\hat h_1,\cdots, 1-\hat h_n)^T \in \ZZ[ab(G)]^n
	\]
	wobei die $\hat h_i \in G$ eindeutig aus den $\hat e_2^i$ mit der obigen Konstruktion hervorgehen und als Restklassen in $F$ betrachtet werden. Offensichtlich gilt jeweils: $\langle \hat g_i \rangle = \langle \hat h_i \rangle = ab(G)$. Die Arbeit ist nun getan, von hier an möchten wir die Topologie vergessen und behandeln unseren Kettenkomplex mit algebraischen Methoden:

	Durch eventuellen Basiswechsel für die Moduln $C_1$ und $C_2$ können wir annehmen, dass bei den Matrixdarstellungen von $\partial_1$ und $\partial_3$ die $g_i=h_i$ übereinstimmen. Da $F=ab(G)=\ZZ^{b_1(M)}$, können wir für diese Basen weiter annehmen, dass $\langle g_1,\cdots,g_{b_1(M)}\rangle=F$ und für alle anderen $g_i$ bei denen $i> b_1(M)$, die Matrixeinträge verschwinden also $g_i=1$. Bezeichne zu dieser Wahl von Basen $M= (m_1,\dots,m_n)$ die darstellende Matrix von $\partial_2$ mit Spalten $m_i\in \ZZ[F]^n$. In den folgenden Matrixberechnungen wird folgende Notation verwendet: $M_{ij}$ bezeichnet die Determinante der $(n-1)\times (n-1)$ Minore die durch Streichen der $i$-ten Zeile und $j$-ten Spalte entsteht, $\underline m_j$ betont das Weglassen dieser Spalte und $m_j^i\in \ZZ[F]^{n-1}$ bezeichnet die Spalte die durch Kürzen um den $i$-ten Eintrag entsteht. Also ist nach dieser Schreibweise $M_{ij}=det(m^i_1, \dots , \underline {m^i_j} , \cdots m^i_n)$. Das Alexander Ideal $I_M(G)$ berechnet sich dann aus $\langle M_{ij} \rangle$.

	Zunächst soll ein Zusammenhang zwischen $M_{ij}$ und $M_{ik}$ hergestellt werden unter Ausnutzung der Beziehung: $m_k(1-g_k)+\sum_{l \neq k}m_l(1-g_l)= \sum m_l(1-g_l)=\partial_3 \circ \partial_2 =0$:
	\begin{align*}
	&M_{ij}(1-g_k)=\\
	&	=(1-g_k)\det(m^i_1,\cdots,\underline {m^i_j},\cdots,m^i_n)&=& \det(m^i_1,\cdots,(1-g_k)m^i_k,\dots,\underline {m^i_j},\cdots)\\
	&				 =\det(m^i_1,\cdots,\sum^i_{l \neq k}m^i_l(1-g_l),\dots,\underline {m^i_j},\cdots) &=&\det((m^i_1,\cdots,-(1-g_j)m^i_j,\dots,\underline {m^i_j},\cdots)\\
	&				 =-(1-g_j)\det((m^i_1,\cdots,m^i_j,\dots,\underline {m^i_j},\cdots) &=& \pm M_{ik}(1-g_j)
	\end{align*}

	Man beachte, dass in der letzten Zeile das $m_j^i$ immer noch an der $k$-ten Stelle steht, also die letzte Gleichheit durch paarweises Vertauschen mit dieser Spalte entsteht. Mit den gleichen Umformungen für Zeilen erhält man ebenso $M_{ij}(1-g_l)=\pm M_{lj}(1-g_i)$. Zusammen ergibt das die Beziehung~\eqref{eq:minors} die im Folgenden ausgiebig betrachtet wird.
	\begin{equation}
		M_{ij}(1-g_k)(1-g_l)=M_{ik}(1-g_j)(1-g_l)=M_{lk}(1-g_j)(1-g_i) \label{eq:minors}
	\end{equation}

	Da $b_1(M)\neq 0$ folgt, dass $(1-g_1)\neq 0$. Somit folgt mit~\eqref{eq:minors}, dass die Determinanten der Minoren verschwinden, wenn sie die erste $b_1(M)\times b_1(M)$ Hauptminore enthalten, aus der maximal eine Zeile \emph{oder} eine Spalte entnommen wurde. Mit anderen Worten, da $\ZZ[F]$ nullteilerfrei ist folgt mit~\eqref{eq:minors}:
	\[
		M_{ij}(1-g_1)^2 = M_{11}*0=0 \implies I_M(G)= \langle M_{ij}|i,j \leq b_1(M) \rangle
	\]
	Seien im Folgenden also stets $i,j\leq b_1(M)$.

	Für die Elemente $M_{ii}$ die aus symmetrischer Kürzung entstehen, liefert~\eqref{eq:minors} die Gleichheit $M_{ii}(1-g_j)^2=\pm M_{jj}(1-g_i)^2$, die nach Annahme an $b_1(M)\geq 2$ nicht-trivial ist. Da aber $\ZZ[F]$ ein faktorieller Ring ist, also die Faktorisierung in Primelemente eindeutig ist und $g_i,g_j$ so gewählt wurden, dass $1-g_i$ und $1-g_j$ koprim sind, ist folgende Wahl gerechtfertigt\footnote{Genaugenommen gilt das in jedem kommutativen Ring mit Eins, da per Definition $$M_{11}=M_{11}((1-g_i)^2x +(1-g_1)^2 y)= M_{ii}(1-g_1)^2x + M_{11}(1-g_1)^2y =(1-g_1)^2 (M_{ii}x+ M_{11} y)$$}:
	\[
		\Delta = \frac{M_{11}}{(1-g_1)^2} = \pm \frac{M_{22}}{(1-g_2)^2} = \cdots = \pm \frac{M_{b_1(M)}}{(1-g_{b_1(M)})^2}
	\]

	Man erhält:
	
	\begin{equation}
		M_{ii}= \pm \Delta (1-g_i)^2 \label{eq:diagonalelements}
	\end{equation}
	Allgemein führt man mit~\eqref{eq:minors} jede Determinante auf~\eqref{eq:diagonalelements} folgendermaßen zurück:
	\begin{align}
		M_{ij} &= \frac{M_{ij}(1-g_1)^2}{(1-g_1)^2} 
				= \pm \frac{M_{11}(1-g_i)(1-g_j)}{(1-g_1)^2}
				= \pm \Delta (1-g_i)(1-g_j)\\
		\implies &I_M(G) = (\Delta) \cdot m(F)^2 
	\end{align}
	wobei das Theorem also für geschlossenes $M$ folgt, da offensichtlich $(\Delta)$ das kleinste Hauptideal ist, das $I_M(G)$ enthält.

	Falls $M$ nun Rand hat, liefert der gleiche Beweis mit etwas Vernunft das gleiche Ergebnis. 
\end{proof} 

Für den Fall, dass $b_1(M)=1$, so wissen wir bereits einfach, dass $I(M)$ unabhängig der beiden angegebenen Definitionen ein Hauptideal ist. Weiter haben wir in diesem Fall bereits gesehen, wie die Alexander-Norm einer Kohomologieklasse $\phi$ in Zusammenhang mit $b_1(\ker \phi)$ steht (falls $b_1(M)=1$ so stimmt die Alexander-Norm mit dem Grad des Alexander Polynoms überein). Letzteres, also einen Beziehung von $b_1(\ker\phi)$ und $||\phi||_A$ soll nun im Fall für $b_1(M)\geq 1$ hergeleitet werden.

\begin{lem}
\label{lem:alexnorm}
	Falls $\Delta_G \neq 0$ mit $G=\pi_1(M)$, so gilt für primitive $\phi \in H^1(M;\ZZ)$ die in einem offenen Kegel durch eine offene berandende Seite des Polytops der Alexander-Einheitskugel liegen:
	\[
		b_1(\ker\phi) = ||\phi||_A + 1 + b_3(M)
	\]
\end{lem}
\begin{proof}
	Zur Berechnung des Alexander Polynoms einer Gruppe verwendet man häufig eine Abbildung auf den maximalen frei abelschen Quotienten, die man auf den ganzzahligen Gruppenringen fortsetzt. Dies ermöglicht einem zwischen verschiedenen Ebenen hin und herzuspringen. Ähnlich soll nun hier $\phi \in H^1(M;\ZZ) \cong \Hom(G,\ZZ) $ fortgesetzt werden: $		\hat \phi:\ZZ[G] \to \laurent \ZZ t$.\\
	Wendet man diesen fortgesetzten Homomorphismus auf $\Delta_G\in \ZZ G$ an erhält man: \[
		\hat \phi \Delta_G = \hat \phi (\sum \lambda_i g_i)= \sum \lambda_i \phi(g_i) = \sum \lambda_i t^{\phi(g_i)} = \sum \hat \lambda_{\phi(g_i)} t^{\phi(g_i)}
	\]
	Die $\hat \lambda_{\phi(g_i)}$ fassen alle Koeffizienten zum gleichen Monom zusammen, sodass in der Summe kein Gruppenelement mehrfach auftritt. Per Definition gilt also:
	\[
		\Grad \phi(\Delta_G) = \sup_{\hat \lambda_{\phi(g_{i/j})} \neq 0} (\phi(g_i)- \phi(g_j))
	\]
	Da nun aber $\phi$ aus einem Kegel durch eine offene Seite des Alexander Polytops gewählt wurde, verschwinden unter $\hat \phi$ keine Koeffizienten die den Grad maximieren, also:
	\[
		\sup_{\hat \lambda_{\phi(g_{i/j})} \neq 0} (\phi(g_i)- \phi(g_j)) = \sup_{\lambda_i}(\phi(g_i)- \phi(g_j))
	\]
	nun folgt das Lemma mit $(\Delta_\phi)=\phi(I(G))=\langle(t-1)^{1+b_1(M)}\hat\phi\Delta_G \rangle	$.
\end{proof}

\subsection{Kombinieren der Ergebnisse}
    
Mit folgendem Beweis endet schließlich dieses Kapitel und somit der Beweis des Haupttheorems.

\begin{proof}[Beweis von Theorem~\ref{thm:haupttheorem}(McMullen)]
	Sei wieder $G=\pi_1(M)$ und ohne Einschränkung $\Delta_G\neq 0$. Definiere weiter \[
	P=\begin{cases}
		b_3(M)+1 &, \text{ falls } b_1(M) \geq 2	\\
		0
	\end{cases}
	\]	
	Nach Corollar~\ref{cor:degreealex} zusammen mit Theorem~\ref{thm:keralexnorm} folgt, dass für alle primitiven $\phi \in H^1(M;\ZZ)$ die in einem offenen Kegel zu einer Seite der Alexander Einheitskugel liegen, die folgende Gleichung:
	
		\begin{equation}
		b_1(\ker\phi)= \alex \phi + P \label{eq:alexnorm}
		\end{equation}
	
	Doch nach Lemma~\ref{lem:norm} definieren sowohl Thurston- auch als Alexander-Norm jeweils Halbnormen, folglich ist es keine Einschränkung die Aussage für diese $\phi$ zu zeigen. Man fixiere also dieses gewählte $\phi$ und wähle gemäß Lemma~\ref{lem:minS} eine zu $\phi$ duale eingebettete orientierte Fläche $S$, mit den geforderten Minimalitätseigenschaften. Aus dem Lemma folgt dann, dass wir eine obere Schranke erhalten:
	\begin{equation}
		b_1(\ker\phi) \leq b_1(S) \label{eq:obereSchrankeS}
	\end{equation}
	Ohne Einschränkung habe $S$ nicht-positive Eulercharakteristik, da sonst wegen $b_1(S)= b_0(S) + b_2(S) - \chi(S) \leq 1$, die obere Schranke für $b_1(\ker\phi)\leq 1$  mit \eqref{eq:alexnorm} impliziert, dass die Alexander-Norm verschwindet (man beachte hierbei und im Folgenden die Fälle $b_2(S)=b_3(M) \in \{0,1\}$). Das Theorem folgt nun aus Zusammensetzung der erarbeiteten Gleichungen:
	
		\begin{align*}
		\thur \phi = -\chi(S) &= b_1(S)-b_0(S)-b_1(S)\\
					&\geq b_1(\ker\phi) -1-b_3(M) &&= \alex \phi +P -b_3(M)-1\\
					&&& = \begin{cases}
						\thur \phi &, \text{falls } b_1(M)\geq 2\\
						\thur \phi -b_3(M)-1
					\end{cases}
		\end{align*}
\end{proof} 
	Die einzige Abschätzung entsteht aus der Wahl der Fläche bei Lemma~\ref{lem:minS} durch $b_1(\ker\phi)\leq b_1(S)$. Dass diese bei einer Faserung verschwindet, wird im nächsten Kapitel diskutiert. Ebenso wird man im nächsten Kapitel einsehen, dass die Abschätzung aus dem Lemma echt ist, also gleichbedeutend, es existieren durchaus Kohomologieklassen mit $\thur \phi > \alex \phi$.